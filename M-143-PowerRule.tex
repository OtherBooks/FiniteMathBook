\documentclass[10pt]{article}

% The following command leaves more space between lines.  That's great
% when correcting drafts.  When you comment it out, however, the
% output looks much nicer.
%
\linespread{1.0}

\usepackage{amsmath}
\usepackage{amssymb}
\usepackage{graphicx}
\usepackage{epsfig}
\usepackage{latexsym}
\usepackage{amsthm}

\usepackage{mathrsfs}

\usepackage{multicol}



\usepackage[colorlinks,citecolor=blue]{hyperref}

\usepackage[latin1]{inputenc}

\usepackage{tikz-cd}
\usepackage{pgfplots}

%\usepackage{3dplot}

\usetikzlibrary{matrix,arrows,decorations.pathmorphing}


\usepackage[scale=0.8]{geometry}


%\usepackage{umoline}\setlength{\UnderlineDepth}{1pt}
%\usepackage[linktocpage=true]{hyperref}

\input xy
\xyoption{all}


%\addtolength{\hoffset}{-.5in}
%\addtolength{\textwidth}{1in}
%\setlength{\parindent}{.5in}
%\setlength{\textheight}{9.5in} \setlength{\topmargin}{-2cm}


\pagestyle{myheadings}\parindent 0em




\usepackage[latin1]{inputenc}


%------------------copy and posted code from the internets-------------

%\numberwithin{equation}{section} % comment out when neccessary

\newtheorem{theorem}[equation]{Theorem}
\newtheorem{lemma}[equation]{Lemma}
\newtheorem{proposition}[equation]{Proposition}
\newtheorem{corollary}[equation]{Corollary}


\theoremstyle{definition}
\newtheorem{definition}[equation]{Definition}
\newtheorem{example}[equation]{Example}
\newtheorem{remark}[equation]{Remark}
\newtheorem{problem}[equation]{Problem}



\newcommand{\R}[1]{\mathbb{R}^{#1}}
\newcommand{\C}[1]{\mathbb{C}^{#1}}
\newcommand{\Z}[1]{\mathbb{Z}^{#1}}
\newcommand{\K}[1]{\mathbb{K}^{#1}}
\newcommand{\embed}[0]{\hookrightarrow}
\newcommand{\TT}[4]{\begin{tabular}{| c | c |}\hline $#1$ & $#2$ \\ \hline $#3$ & $#4$ \\ \hline\end{tabular}} %goddamn it
\newcommand{\partd}[2]{\frac{\partial #1}{\partial #2}}
\newcommand{\limit}[2]{\displaystyle{ \lim_{#1 \to #2}}}
\newcommand{\vectornorm}[1]{\left|\left|#1\right|\right|}
\newcommand{\Ker}[0]{\text{\textnormal{Ker}}}
\newcommand{\Hom}[0]{\text{\textnormal{Hom}}}
\newcommand{\circled}[1]{\tikz[baseline=(char.base)]{
            \node[shape=circle,draw,inner sep=2pt] (char) {#1};}}


\newcommand{\T}{\rotatebox[origin=c]{180}{$\scriptscriptstyle \perp $}}
\newcommand{\x}{\textbf{x}}
\newcommand{\y}{\textbf{y}}
\newcommand{\supp}{\text{\textnormal{supp}}}
\newcommand{\csupp}{\text{\textnormal{cosupp}}}
\newcommand{\found}{\text{\textnormal{found}}}
\newcommand{\roof}{\text{\textnormal{roof}}}

\newcommand{\bcup}{\displaystyle\bigcup}
\newcommand{\bcap}{\displaystyle\bigcap}
\newcommand{\dsum}{\displaystyle\sum}
\newcommand{\dint}{\displaystyle\int}





\begin{document}
%

{\bf Name:} \hrulefill\hrulefill\hrulefill\\
{\bf M143} \qquad \qquad \\
{\bf Basic and Power rule of Derivatives}\\ %(look familiar??)\\
%Show all work for full/partial credit.
%---------------- End of the document ---------------

\section{Basic rules of the derivative}

At this point we probably are, or should be, sick of computing limits like: $$\limit{h}{0}\frac{2(x+h)^3-4(x+h)-(2x^3-4x)}{h}.$$  This is technically the derivative of $f(x)=2x^3-4x$.  But there's 2 sides to mathematics, theres the formal definition, and then we should leverage our knowledge and try to find shortcuts to some of these things.  So let's  figure some of these things out.\\

Given functions $f(x), g(x)$ and constants $a,b$ we have: $$\frac{d}{dx}[a(f)x)+bg(x)]=af'(x)+bg'(x).$$  How do we know that?  Well:

\begin{eqnarray*}
\frac{d}{dx}[a(f)x)+bg(x)]&=&\limit{h}{0}\frac{af(x+h)+bg(x+h)-(af(x)-bg(x))}{h}\\
&=&\limit{h}{0}\frac{af(x+h)-af(x)}{h}+\limit{h}{0}\frac{bg(x+h)-bg(x)}{h}\\
&=&a\limit{h}{0}\frac{f(x+h)-f(x)}{h}+b\limit{h}{0}\frac{g(x+h)-g(x)}{h}\\
&=&af'(x)+bg'(x).
\end{eqnarray*}

\section{Power Rule}

We may have noticed a pattern with derivatives.  But to illustrate it explicitly:

\begin{itemize}
\item The derivative of $f(x)=x^0$, it's the derivative of the  constant function $f(x)=1$, and since its a horizontal line, it never has any slope and $f'(x)=0$.
\item  The derivative of $f(x)=x^1$, it's the derivative of the line $f(x)=x$ which always has slope 1, so $f'(x)=1$.

\item The derivative of $f(x)=x^2$ is:

\begin{eqnarray*}
f(x)&=&\limit{h}{0}\frac{f(x+h)-f(x)}{h}\\
&=&\limit{h}{0} \frac{(x+h)^2-x^2}{h}\\
&=&\limit{h}{0} \frac{x^2+2xh+h^2-x^2}{h}\\
&=&\limit{h}{0} \frac{(2x+h)h}{h}\\
&=&\limit{h}{0} 2x+h=2x.\\
\end{eqnarray*}


\item The derivative of $f(x)=x^3$ is:

\begin{eqnarray*}
f(x)&=&\limit{h}{0}\frac{f(x+h)-f(x)}{h}\\
&=&\limit{h}{0} \frac{(x+h)^3-x^3}{h}\\
&=&\limit{h}{0} \frac{x^3+3x^2h+3xh^2+h^3-x^3}{h}\\
&=&\limit{h}{0} \frac{(3x^2+3xh+h^2)h}{h}\\
&=&\limit{h}{0} 3x^2+3xh+h^2=3x^2.\\
\end{eqnarray*}
 
\end{itemize}

To generalize this, we have a rule of differentiation called the {\bf power rule} which goes: $\frac{d}{dx}[x^n]=nx^{n-1}$. \\ 

To verify:

\begin{itemize}
\item $\frac{d}{dx}[x^0]=0x^{0-1}=0$.
\item $\frac{d}{dx}[x^1]=1x^{1-1}=1$.
\item $\frac{d}{dx}[x^2]=2x^{2-1}=2x$.
\item $\frac{d}{dx}[x^3]=3x^{3-1}=3x^2$.
\end{itemize}

What's fascinating is that this rule applies to {\bf all} powers, not just integer powers.


\section{Examples}

We can combine these rules to find the derivatives of some basic functions.  So consider the derivative of $f(x)=x^4-2x^2+7x-3$.  It would be:

\begin{eqnarray*}
f'(x)&=&\frac{d}{dx}[x^4-2x^2+7x-3]\\
&=&\frac{d}{dx}[x^4-2x^2+7x^1-3x^0]\\
&=&4x^3-2(2x^1)+7(1x^0)-3(0x^{-1})\\
&=&12x^3-4x+7.
\end{eqnarray*}

Whereas the derivative of $g(x)=\sqrt{4x}+\frac{3}{x}-\frac{2}{x^2}$ would be:

\begin{eqnarray*}
g'(x)&=&\frac{d}{dx}[\sqrt{4x}+\frac{3}{x}-\frac{2}{x^2}]\\
&=&\frac{d}{dx}[2x^{1/2}+3x^{-1}-2x^{-2}]\\
&=&2(1/2)x^{-1/2}+3(-1x^{-2})-2(2x^{-3})\\
&=&x^{-1/2}-3x^{-2}+4x^{-3}\\
&=&\frac{1}{\sqrt{x}}-\frac{3}{x^2}+\frac{4}{x^3}.
\end{eqnarray*}
























\end{document}
