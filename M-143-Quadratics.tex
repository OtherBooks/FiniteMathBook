\documentclass[10pt]{article}

% The following command leaves more space between lines.  That's great
% when correcting drafts.  When you comment it out, however, the
% output looks much nicer.
%
\linespread{1.0}

\usepackage{amsmath}
\usepackage{amssymb}
\usepackage{graphicx}
\usepackage{epsfig}
\usepackage{latexsym}
\usepackage{amsthm}

\usepackage{mathrsfs}

\usepackage{multicol}



\usepackage[colorlinks,citecolor=blue]{hyperref}

\usepackage[latin1]{inputenc}

\usepackage{tikz-cd}
\usepackage{pgfplots}

%\usepackage{3dplot}

\usetikzlibrary{matrix,arrows,decorations.pathmorphing}


\usepackage[scale=0.8]{geometry}


%\usepackage{umoline}\setlength{\UnderlineDepth}{1pt}
%\usepackage[linktocpage=true]{hyperref}

\input xy
\xyoption{all}


%\addtolength{\hoffset}{-.5in}
%\addtolength{\textwidth}{1in}
%\setlength{\parindent}{.5in}
%\setlength{\textheight}{9.5in} \setlength{\topmargin}{-2cm}


\pagestyle{myheadings}\parindent 0em




\usepackage[latin1]{inputenc}


%------------------copy and posted code from the internets-------------

%\numberwithin{equation}{section} % comment out when neccessary

\newtheorem{theorem}[equation]{Theorem}
\newtheorem{lemma}[equation]{Lemma}
\newtheorem{proposition}[equation]{Proposition}
\newtheorem{corollary}[equation]{Corollary}


\theoremstyle{definition}
\newtheorem{definition}[equation]{Definition}
\newtheorem{example}[equation]{Example}
\newtheorem{remark}[equation]{Remark}
\newtheorem{problem}[equation]{Problem}



\newcommand{\R}[1]{\mathbb{R}^{#1}}
\newcommand{\C}[1]{\mathbb{C}^{#1}}
\newcommand{\Z}[1]{\mathbb{Z}^{#1}}
\newcommand{\K}[1]{\mathbb{K}^{#1}}
\newcommand{\embed}[0]{\hookrightarrow}
\newcommand{\TT}[4]{\begin{tabular}{| c | c |}\hline $#1$ & $#2$ \\ \hline $#3$ & $#4$ \\ \hline\end{tabular}} %goddamn it
\newcommand{\partd}[2]{\frac{\partial #1}{\partial #2}}
\newcommand{\limit}[2]{\displaystyle{ \lim_{#1 \to #2}}}
\newcommand{\vectornorm}[1]{\left|\left|#1\right|\right|}
\newcommand{\Ker}[0]{\text{\textnormal{Ker}}}
\newcommand{\Hom}[0]{\text{\textnormal{Hom}}}
\newcommand{\circled}[1]{\tikz[baseline=(char.base)]{
            \node[shape=circle,draw,inner sep=2pt] (char) {#1};}}


\newcommand{\T}{\rotatebox[origin=c]{180}{$\scriptscriptstyle \perp $}}
\newcommand{\x}{\textbf{x}}
\newcommand{\y}{\textbf{y}}
\newcommand{\supp}{\text{\textnormal{supp}}}
\newcommand{\csupp}{\text{\textnormal{cosupp}}}
\newcommand{\found}{\text{\textnormal{found}}}
\newcommand{\roof}{\text{\textnormal{roof}}}

\newcommand{\bcup}{\displaystyle\bigcup}
\newcommand{\bcap}{\displaystyle\bigcap}
\newcommand{\dsum}{\displaystyle\sum}
\newcommand{\dint}{\displaystyle\int}





\begin{document}
%

{\bf Name:} \hrulefill\hrulefill\hrulefill\\
{\bf M143} \qquad \qquad \\
{\bf Quadratic Functions}\\ %(look familiar??)\\
%Show all work for full/partial credit.
%---------------- End of the document ---------------

\section{Form of Quadratic functions}

\begin{definition}
A function $q(x)$ is a {\bf quadratic function} if it may be written as: $$q(x)=ax^2+bx+c.$$ where $a\neq0$.
\end{definition}

The classic example is the standard parabola,  $y=x^2$ or a quadratic with $a=1. b,c=0$.  \url{https://www.desmos.com/calculator/aktz8g8cbh}.  We should be familiar with some of the properties of this shape, it's symmetric on both sides and it goes to positive  infinity in both directions (as $x^2$ cannot be negative).\\

It would be nice if we could show that all quadratics had the same essential shape.  To do this, we must first show that all quadratics also have the form $$q(x)=a(x-h)^2+k.$$

To see this:

\begin{eqnarray*}
q(x)&=&ax^2+bx+c\\
&=&a(x^2+\frac{b}{a}x)+c\\
&=&a(x^2+2\frac{b}{2a}x+\frac{b^2}{4a^2})+c-\frac{b^2}{4a}\\
&=&a(x+\frac{b}{2a})^2+c-\frac{b^2}{4a}\\
\end{eqnarray*}
So by letting $h=-\frac{b}{2a}$ and $k=c-\frac{b^2}{4a}$, we have our form.\\

An interesting byproduct of this is that if we were to solve $ax^2+bx+c=0$, we would get:

\begin{eqnarray*}
ax^2+bx+c&=&0\\
a(x+\frac{b}{2a})^2+c-\frac{b^2}{4a}&=&0\\
a(x+\frac{b}{2a})^2=\frac{b^2}{4a}-c\\
a(x+\frac{b}{2a})^2=\frac{b^2-4ac}{4a}\\
(x+\frac{b}{2a})^2=\frac{b^2-4ac}{4a^2}\\
x+\frac{b}{2a}=\frac{\pm\sqrt{b^2-4ac}}{2a}\\
x=\frac{-b\pm\sqrt{b^2-4ac}}{2a}\\
\end{eqnarray*}

Which is the quadratic formula.

\section{Parabolas}

Now that we have this form $a(x-h)^2+k$, consider how $a,h,$ and $k$  affect the shape of the graph: \url{https://www.desmos.com/calculator/4zqd9chlna}.\\

Sliding $k$ shifts the parabola up and down.  We can think of this as adding to or detracting from the value of $x^2$.  Sliding $h$ shifts the graph left and right.  We can think of this as slowing down or speeding up $x^2$ by $h$ units.  $a$ stretches or squashes $x^2$, or if $a$ is negative, also reflects this across the $x$ axis.  But all of these transformations preserve the fundamental shape of a parabola, so all quadratic functions have this shape.









\end{document}
