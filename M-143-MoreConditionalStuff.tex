\documentclass[10pt]{article}

% The following command leaves more space between lines.  That's great
% when correcting drafts.  When you comment it out, however, the
% output looks much nicer.
%
\linespread{1.0}

\usepackage{amsmath}
\usepackage{amssymb}
\usepackage{graphicx}
\usepackage{epsfig}
\usepackage{latexsym}
\usepackage{amsthm}

\usepackage{mathrsfs}

\usepackage{multicol}



\usepackage[colorlinks,citecolor=blue]{hyperref}

\usepackage[latin1]{inputenc}

\usepackage{tikz-cd}
\usepackage{pgfplots}

%\usepackage{3dplot}

\usetikzlibrary{matrix,arrows,decorations.pathmorphing}
\usepackage{circuitikz}

\usepackage[scale=0.8]{geometry}


%\usepackage{umoline}\setlength{\UnderlineDepth}{1pt}
%\usepackage[linktocpage=true]{hyperref}

\input xy
\xyoption{all}


%\addtolength{\hoffset}{-.5in}
%\addtolength{\textwidth}{1in}
%\setlength{\parindent}{.5in}
%\setlength{\textheight}{9.5in} \setlength{\topmargin}{-2cm}


\pagestyle{myheadings}\parindent 0em




\usepackage[latin1]{inputenc}


%------------------copy and posted code from the internets-------------

%\numberwithin{equation}{section} % comment out when neccessary

\newtheorem{theorem}[equation]{Theorem}
\newtheorem{lemma}[equation]{Lemma}
\newtheorem{proposition}[equation]{Proposition}
\newtheorem{corollary}[equation]{Corollary}


\theoremstyle{definition}
\newtheorem{definition}[equation]{Definition}
\newtheorem{example}[equation]{Example}
\newtheorem{remark}[equation]{Remark}
\newtheorem{problem}[equation]{Problem}



\newcommand{\R}[1]{\mathbb{R}^{#1}}
\newcommand{\C}[1]{\mathbb{C}^{#1}}
\newcommand{\Z}[1]{\mathbb{Z}^{#1}}
\newcommand{\K}[1]{\mathbb{K}^{#1}}
\newcommand{\embed}[0]{\hookrightarrow}
\newcommand{\TT}[4]{\begin{tabular}{| c | c |}\hline $#1$ & $#2$ \\ \hline $#3$ & $#4$ \\ \hline\end{tabular}} %goddamn it
\newcommand{\partd}[2]{\frac{\partial #1}{\partial #2}}
\newcommand{\limit}[2]{\displaystyle{ \lim_{#1 \to #2}}}
\newcommand{\vectornorm}[1]{\left|\left|#1\right|\right|}
\newcommand{\Ker}[0]{\text{\textnormal{Ker}}}
\newcommand{\Hom}[0]{\text{\textnormal{Hom}}}
\newcommand{\circled}[1]{\tikz[baseline=(char.base)]{
            \node[shape=circle,draw,inner sep=2pt] (char) {#1};}}


\newcommand{\T}{\rotatebox[origin=c]{180}{$\scriptscriptstyle \perp $}}
\newcommand{\x}{\textbf{x}}
\newcommand{\y}{\textbf{y}}
\newcommand{\supp}{\text{\textnormal{supp}}}
\newcommand{\csupp}{\text{\textnormal{cosupp}}}
\newcommand{\found}{\text{\textnormal{found}}}
\newcommand{\roof}{\text{\textnormal{roof}}}

\newcommand{\bcup}{\displaystyle\bigcup}
\newcommand{\bcap}{\displaystyle\bigcap}
\newcommand{\dsum}{\displaystyle\sum}
\newcommand{\dint}{\displaystyle\int}





\begin{document}
%

{\bf Name:} \hrulefill\hrulefill\hrulefill\\
{\bf M143} \qquad \qquad \\
{\bf More Conditional Stuff}\\ %(look familiar??)\\
%Show all work for full/partial credit.
%---------------- End of the document ---------------



\section{More stuff on Conditional Statements}

There are some common misinterpretations of the conditional statement that should be addressed.  In particular, there are colloquial statements which people believe that are equivalent to the conditional which are not.\\

Given the statement $p\to q$, is it equivalent to say $q\to p$?  (We say that $q\to p$ is the \textbf{converse} of $p\to q$.)  We can check with truth tables, but before that, let's ask ourselves this:  Suppose that $p=$``I drink bleach" and $q=$``I get sick."  I think we can all buy that $p\to q$ i.e.\ if I drink bleach then I will get sick.  Now the converse, $q\to p$ is the statement ``If I get sick, then I drank bleach".  Is this the same statement?\\

Doesn't seem so, you could have gotten sick from any sort of things, it didn't have to be bleach.  Another example is if $p=$``I'm a square" and $q=$``I'm a rectangle."  All squares are rectangles, not all rectangles are squares.  So if $p\to q$, we can think of that as all the circumstances where $p$ is true lives inside the collection of circumstances where $q$ is true, and that the reverse is not the same thing.  And of course:

$$\begin{array}{c|c|c|c}
p & q & p\to q & q\to p\\
\hline
T & T & T & T\\
T & F & F & T\\
F & T & T & F \\
F & F & T & T
\end{array}$$


So these columns are not the same and so the statements are not equal.\\

What about $\sim p \to \sim q$? (This is the {\bf inverse} of $p\to q$)  Well this would be ``If I'm not drinking bleach then I won't get sick" or ``If I'm not a square then I'm not a rectangle."  This doesn't seem like it's true either.  You can not drink bleach, but then drink paint thinner instead and you'd be sick, and of course there are plenty of rectangles that aren't squares.  That and:

$$\begin{array}{c|c|c|c|c|c}
p & q & p\to q &\sim p & \sim q& \sim p\to \sim q\\
\hline
T & T & T & F & F & T\\
T & F & F & F & T &T\\
F & T & T & T & F & F \\
F & F & T & T & T & T
\end{array}$$


What about $\sim q \to \sim p$ (the {\bf contrapositive"} of $p\to q$)?  In our example, that would be, ``If I didn't get sick then I didn't drink bleach" or ``If I'm not a rectangle then I'm not a square".  These DO seem true.  If you didn't get sick you couldn't have drank bleach, if you're not a rectangle, there's no way you could be a square.  Rather than use truth tables, consider:

\begin{eqnarray*}
\sim q \to \sim p&=&(\sim \sim q \vee \sim p)\\
&=&q \vee \sim p\\
&=&\sim p \vee q\\
&=&p\to q.
\end{eqnarray*}

So these things ARE the same.


\section{Biconditional}

Maybe you want to describe 2 statements where $p\to q$ and $q\to p$, i.e. whenever 1 is true then the other is true.  Well this is literally $(p\to q)\wedge (q\to p)$.



$$\begin{array}{c|c|c|c|c}
p & q & p\to q & q\to p& (p\to q)\wedge (q\to p)\\
\hline
T & T & T & T&T\\
T & F & F & T&F\\
F & T & T & F &F\\
F & F & T & T&T
\end{array}$$



We symbolize this with $p\leftrightarrow q:$



$$\begin{array}{c|c|c}
p & q & p\leftrightarrow q \\
\hline
T & T & T \\
T & F & F\\
F & T & F \\
F & F & T 
\end{array}$$








































\end{document}
