\documentclass[10pt]{article}

% The following command leaves more space between lines.  That's great
% when correcting drafts.  When you comment it out, however, the
% output looks much nicer.
%
\linespread{1.0}

\usepackage{amsmath}
\usepackage{amssymb}
\usepackage{graphicx}
\usepackage{epsfig}
\usepackage{latexsym}
\usepackage{amsthm}

\usepackage{mathrsfs}

\usepackage{multicol}



\usepackage[colorlinks,citecolor=blue]{hyperref}

\usepackage[latin1]{inputenc}

\usepackage{tikz-cd}
\usepackage{pgfplots}

%\usepackage{3dplot}

\usetikzlibrary{matrix,arrows,decorations.pathmorphing}
\usepackage{circuitikz}

\usepackage[scale=0.8]{geometry}


%\usepackage{umoline}\setlength{\UnderlineDepth}{1pt}
%\usepackage[linktocpage=true]{hyperref}

\input xy
\xyoption{all}


%\addtolength{\hoffset}{-.5in}
%\addtolength{\textwidth}{1in}
%\setlength{\parindent}{.5in}
%\setlength{\textheight}{9.5in} \setlength{\topmargin}{-2cm}


\pagestyle{myheadings}\parindent 0em




\usepackage[latin1]{inputenc}


%------------------copy and posted code from the internets-------------

%\numberwithin{equation}{section} % comment out when neccessary

\newtheorem{theorem}[equation]{Theorem}
\newtheorem{lemma}[equation]{Lemma}
\newtheorem{proposition}[equation]{Proposition}
\newtheorem{corollary}[equation]{Corollary}


\theoremstyle{definition}
\newtheorem{definition}[equation]{Definition}
\newtheorem{example}[equation]{Example}
\newtheorem{remark}[equation]{Remark}
\newtheorem{problem}[equation]{Problem}



\newcommand{\R}[1]{\mathbb{R}^{#1}}
\newcommand{\C}[1]{\mathbb{C}^{#1}}
\newcommand{\Z}[1]{\mathbb{Z}^{#1}}
\newcommand{\K}[1]{\mathbb{K}^{#1}}
\newcommand{\embed}[0]{\hookrightarrow}
\newcommand{\TT}[4]{\begin{tabular}{| c | c |}\hline $#1$ & $#2$ \\ \hline $#3$ & $#4$ \\ \hline\end{tabular}} %goddamn it
\newcommand{\partd}[2]{\frac{\partial #1}{\partial #2}}
\newcommand{\limit}[2]{\displaystyle{ \lim_{#1 \to #2}}}
\newcommand{\vectornorm}[1]{\left|\left|#1\right|\right|}
\newcommand{\Ker}[0]{\text{\textnormal{Ker}}}
\newcommand{\Hom}[0]{\text{\textnormal{Hom}}}
\newcommand{\circled}[1]{\tikz[baseline=(char.base)]{
            \node[shape=circle,draw,inner sep=2pt] (char) {#1};}}


\newcommand{\T}{\rotatebox[origin=c]{180}{$\scriptscriptstyle \perp $}}
\newcommand{\x}{\textbf{x}}
\newcommand{\y}{\textbf{y}}
\newcommand{\supp}{\text{\textnormal{supp}}}
\newcommand{\csupp}{\text{\textnormal{cosupp}}}
\newcommand{\found}{\text{\textnormal{found}}}
\newcommand{\roof}{\text{\textnormal{roof}}}

\newcommand{\bcup}{\displaystyle\bigcup}
\newcommand{\bcap}{\displaystyle\bigcap}
\newcommand{\dsum}{\displaystyle\sum}
\newcommand{\dint}{\displaystyle\int}





\begin{document}
%

{\bf Name:} \hrulefill\hrulefill\hrulefill\\
{\bf M143} \qquad \qquad \\
{\bf Quantifiers}\\ %(look familiar??)\\
%Show all work for full/partial credit.
%---------------- End of the document ---------------


\section{Quantifiers}

There are a lot of things in the world.  A lot of people, a lot of animals, a lot of numbers, a lot of toppings available for pizza, a lot of things I could add to this list.  Presumably, sometimes we'd want to make statements about all these things, and sometimes we want to make statements about some of these things.\\

Suppose $x$ is an MSUB student and $I(x)$ is the statement ``$x$ likes ice cream."  We want to perhaps say something more subtle than this. So we introduce the quantifiers $\forall$ and $\exists$.\\

The $\forall$ is pronounced ``for all" or ``for every", indicates that the following statement is true for all values of $x$.  So $\forall I(x)$ would denote ``For every MSUB student $x$, $x$ likes Ice Cream.".  In order for a $\forall$ statement to be true, it has to be true for EVERY possible value of $x$.\\

The $\exists$ pronounced ``there exists" or ``there is" indicates the following statement is true for some values of $x$.  It could be true for 1 of them, or a bunch of them, or for all of them.  The only thing we know for sure is that there is AT LEAST 1 value of $x$ for which it is true.  So $\exists I(x)$ would mean ``There is an MSUB student who likes Ice Cream." or ``At least one MSUB student likes Ice Cream."\\

Let's let $M(x)$ be ``$x$ is a Math Major", $C(x)$ be ``$x$ takes Calculus".  Then consider the following equivalences:

$$
\begin{tabular}{c|c}
{\bf Logic} & {\bf English}\\
\hline
$\forall (M(x)\to C(x))$& For every MSUB student, if you're a Math Major then you take Calculus.\\
$\exists (\sim M(x)\wedge \sim C(x))$ & There is an MSUB student who isn't a Math Major and doesn't take Calculus.\\
$\forall \sim C(x)$ & Every MSUB student does not take calculus.
\end{tabular}
$$

Putting aside whether or not the above statements are actually true, here we just see how we can translate statements from english to logical symbols with quantifiers and back.\\

Also, whenever you see a statement $\forall p(x)$, no matter what $p(x)$ is, it is fair to deduce $\exists p(x)$ automatically.  If somethings true for EVERY $x$, then it also has to be true for at least 1 $x$.  The converse is obviously NOT true.  If something is true for a single $x$, it doesn't mean it will be true for more.


\section{Negating Quantifiers}

Let's consider $\forall I(x)$ or ``Every MSUB student likes Ice Cream."   If this were NOT true, what would that mean?  It would mean that there is some poor soul out on our campus who does not like Ice Cream.  In other words: $\exists \sim I(x)$.\\

What would the negation of $\exists I(x)$ be?  The statement is saying ``There is a student who likes Ice Cream."  What is the opposite of this?  It would mean that every single student doesn't like Ice Cream, i.e.\ $\forall \sim I(x)$.  So our negation rules are:

\begin{eqnarray*}
\sim(\forall p(x))&=&\exists \sim p(x)\\
\sim(\exists p(x))&=&\forall \sim p(x).
\end{eqnarray*}

So let's practice on the statements we made earlier:

\begin{itemize}
\item The negation of ``For every MSUB student, if you're a Math Major then you take Calculus." is:

\begin{eqnarray*}
\sim(\forall (M(x)\to C(x)))&=&\exists\sim (M(x)\to C(x))\\
&=&\exists\sim (\sim M(x)\vee C(x))\\
&=&\exists(M(x)\wedge \sim C(x))\\
\end{eqnarray*}
``At MSUB, there is a student who is a Math Major and doesn't take Calculus."

\item The negation of ``There is an MSUB student who isn't a Math Major and doesn't take Calculus." is:

\begin{eqnarray*}
\sim(\exists (\sim M(x)\wedge \sim C(x)))&=&\forall \sim(\sim M(x)\wedge \sim C(x))\\
&=&\forall M(x)\vee C(x)
\end{eqnarray*}

``Every student at MSUB either is a Math Major or takes Calculus."

\item The negation of ``Every MSUB student does not take calculus." is:

\begin{eqnarray*}
\sim(\forall \sim C(x))&=&\exists \sim\sim C(x)\\
&=&\exists C(x)
\end{eqnarray*}
``There is an MSUB student taking Calculus."






\end{itemize}
Again, I'm not claiming any of these things are true IRL.  But given the original statements, these should read and feel like real opposites.









\end{document}
