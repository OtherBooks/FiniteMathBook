\documentclass[10pt]{article}

% The following command leaves more space between lines.  That's great
% when correcting drafts.  When you comment it out, however, the
% output looks much nicer.
%
\linespread{1.0}

\usepackage{amsmath}
\usepackage{amssymb}
\usepackage{graphicx}
\usepackage{epsfig}
\usepackage{latexsym}
\usepackage{amsthm}

\usepackage{mathrsfs}

\usepackage{multicol}



\usepackage[colorlinks,citecolor=blue]{hyperref}

\usepackage[latin1]{inputenc}

\usepackage{tikz-cd}
\usepackage{pgfplots}

%\usepackage{3dplot}

\usetikzlibrary{matrix,arrows,decorations.pathmorphing}


\usepackage[scale=0.8]{geometry}


%\usepackage{umoline}\setlength{\UnderlineDepth}{1pt}
%\usepackage[linktocpage=true]{hyperref}

\input xy
\xyoption{all}


%\addtolength{\hoffset}{-.5in}
%\addtolength{\textwidth}{1in}
%\setlength{\parindent}{.5in}
%\setlength{\textheight}{9.5in} \setlength{\topmargin}{-2cm}


\pagestyle{myheadings}\parindent 0em




\usepackage[latin1]{inputenc}


%------------------copy and posted code from the internets-------------

%\numberwithin{equation}{section} % comment out when neccessary

\newtheorem{theorem}[equation]{Theorem}
\newtheorem{lemma}[equation]{Lemma}
\newtheorem{proposition}[equation]{Proposition}
\newtheorem{corollary}[equation]{Corollary}


\theoremstyle{definition}
\newtheorem{definition}[equation]{Definition}
\newtheorem{example}[equation]{Example}
\newtheorem{remark}[equation]{Remark}
\newtheorem{problem}[equation]{Problem}



\newcommand{\R}[1]{\mathbb{R}^{#1}}
\newcommand{\C}[1]{\mathbb{C}^{#1}}
\newcommand{\Z}[1]{\mathbb{Z}^{#1}}
\newcommand{\K}[1]{\mathbb{K}^{#1}}
\newcommand{\embed}[0]{\hookrightarrow}
\newcommand{\TT}[4]{\begin{tabular}{| c | c |}\hline $#1$ & $#2$ \\ \hline $#3$ & $#4$ \\ \hline\end{tabular}} %goddamn it
\newcommand{\partd}[2]{\frac{\partial #1}{\partial #2}}
\newcommand{\limit}[2]{\displaystyle{ \lim_{#1 \to #2}}}
\newcommand{\vectornorm}[1]{\left|\left|#1\right|\right|}
\newcommand{\Ker}[0]{\text{\textnormal{Ker}}}
\newcommand{\Hom}[0]{\text{\textnormal{Hom}}}
\newcommand{\circled}[1]{\tikz[baseline=(char.base)]{
            \node[shape=circle,draw,inner sep=2pt] (char) {#1};}}


\newcommand{\T}{\rotatebox[origin=c]{180}{$\scriptscriptstyle \perp $}}
\newcommand{\x}{\textbf{x}}
\newcommand{\y}{\textbf{y}}
\newcommand{\supp}{\text{\textnormal{supp}}}
\newcommand{\csupp}{\text{\textnormal{cosupp}}}
\newcommand{\found}{\text{\textnormal{found}}}
\newcommand{\roof}{\text{\textnormal{roof}}}

\newcommand{\bcup}{\displaystyle\bigcup}
\newcommand{\bcap}{\displaystyle\bigcap}
\newcommand{\dsum}{\displaystyle\sum}
\newcommand{\dint}{\displaystyle\int}





\begin{document}
%

{\bf Name:} \hrulefill\hrulefill\hrulefill\\
{\bf M143} \qquad \qquad \\
{\bf Sums and Products of Matrices }\\ %(look familiar??)\\
%Show all work for full/partial credit.
%---------------- End of the document ---------------


\section{Products of Matrices}

I'm going to do something a little unorthodox here and discuss products before sums.

\begin{example}

Suppose that  company $A$ is sponsoring a fundraising event for a university, and for every Female participant, they donate 2 dollars to charity $X$, 1 dollar to charity $Y$ and 2 dollars to charity $Z$.  For every Male student They donate 1 dollar to charity $X$, 2 dollars to $Y$ and 2 dollars to $Z$.

\begin{enumerate}
\item If 20 Female students and 15 Male students participate, how much money does company $A$ raise for each charity?
\item If $F$ Female students and $M$ Male students participate, how much money does company $A$ raise for each charity?
\end{enumerate}


\begin{enumerate}
\item So we note that Charity $X$ will receive $2(20)+1(15)=55$ dollars.  Charity $Y$ will receive $1(20)+2(15)=50$ dollars, and Charity $Z$ will receive $2(20)+2(15)=70$ dollars.

\item Similarly, Charity $X$ will receive $2F+M$ dollars.  Charity $Y$ will receive $F+2M$ dollars, and Charity $Z$ will receive $2F+2M$ dollars.

\end{enumerate}
\end{example}



Now, we can express the number of Female and Male students as a vector matrix: $\begin{pmatrix} F \\ M\end{pmatrix}$.  Similarly, we can express the number of dollars received by each charity in the same way: $\begin{pmatrix} A\\ B\\ C \end{pmatrix}$.  We would like to express the process of taking $\begin{pmatrix} F \\ M\end{pmatrix}$ and achieving $\begin{pmatrix} A\\ B\\ C \end{pmatrix}$

So we express the Product of matrices as taking the rows of  the left matrix, and ``applying" them to the columns of the right:

$$\begin{pmatrix} a&b \\ c&d\end{pmatrix}\begin{pmatrix} w & x \\ y & z\end{pmatrix}=\begin{pmatrix}aw+by & ax+bz \\ cw+dy & cx+dz  \end{pmatrix}.$$

In order for this product to be defined, the length of the rows on the left matrix has to be the same size as the length of the columns on the right.  Think of it this way, a matrix with $m$ rows and $n$ columns takes in vectors of length $n$ and spits out vectors of length $m$.  In order for that output to be then fed into another such matrix, it would HAVE to have $m$ columns.\\

So with this in mind, consider:

$$A=\begin{pmatrix} 2 & 1 \\ 1 & 2 \\ 2 & 2\end{pmatrix}$$

So: $$\begin{pmatrix} 2 & 1 \\ 1 & 2 \\ 2 & 2\end{pmatrix}\begin{pmatrix} 20 \\ 15\end{pmatrix}=\begin{pmatrix} 2(20)+15 \\ 15+2(15) \\ 2(20)+2(15)\end{pmatrix}=\begin{pmatrix} 55\\50\\70\end{pmatrix}$$

and also:

$$\begin{pmatrix} 2 & 1 \\ 1 & 2 \\ 2 & 2\end{pmatrix}\begin{pmatrix} F \\ M\end{pmatrix}=\begin{pmatrix} 2F+M \\ F+2M\\ 2F+2M\end{pmatrix}$$

\begin{example}
Suppose that for each dollar charity $X$ receives, 80 cents go to meals for the Elderly, and 20 cents go to Homeless shelters, for $Y$ half go to Elderly, half to Homeless, and Charity $Z$ donates all it's money to Homeless shelters.
\begin{enumerate}
\item If each charity receives 100 dollars, how much goes to Elderly, how much to Homeless?
\item If each charity receives $X, Y$ or $Z$ dollars, how much goes to Elderly, how much to Homeless?
\item The key question: if $F$ females and $M$ males attend this fundraiser, how much money would go to Elderly, and how much to Homeless?
\end{enumerate}


\begin{enumerate}
\item Consider this matrix: $$\begin{pmatrix} 0.8 & 0.5 & 0 \\ 0.2 & 0.5 & 1 \end{pmatrix}$$  If We take the product:

$$\begin{pmatrix} 0.8 & 0.5 & 0 \\ 0.2 & 0.5 & 1 \end{pmatrix}\begin{pmatrix} 100 \\ 100 \\ 100\end{pmatrix}=\begin{pmatrix} 80+50 \\ 20+50+100\end{pmatrix}=\begin{pmatrix} 130 \\ 170\end{pmatrix}$$

So \$130 to Elderly and \$170 to Homeless.

\item So in general:

$$\begin{pmatrix} 0.8 & 0.5 & 0 \\ 0.2 & 0.5 & 1 \end{pmatrix}\begin{pmatrix}  X\\Y\\Z \end{pmatrix}= \begin{pmatrix} 0.8X+0.5Y \\ 0.2X+0.5Y+Z\end{pmatrix}$$

\item Putting this all together, consider:

\begin{eqnarray*}
\begin{pmatrix} 0.8 & 0.5 & 0 \\ 0.2 & 0.5 & 1 \end{pmatrix} \begin{pmatrix} 2 & 1 \\ 1 & 2 \\ 2 & 2\end{pmatrix} \begin{pmatrix} F\\M \end{pmatrix}&=&\begin{pmatrix} 0.8(2)+(0.5)1+0(2) & 0.8(1)+0.5(2)+0(2) \\ 0.2(2)+0.5(1)+1(2) & 0.2(1)+0.5(2)+1(2) \end{pmatrix}\begin{pmatrix} F\\M \end{pmatrix}\\
&=&\begin{pmatrix} 2.1 & 1.8 \\ 2.9 & 3.2\end{pmatrix}\begin{pmatrix} F\\M \end{pmatrix}\\
&=&\begin{pmatrix} 2.1F + 1.8M \\ 2.9F + 3.2M\end{pmatrix}
\end{eqnarray*}

\end{enumerate}
\end{example}




\section{Sums of Matrices}

Matrix sums are much more straight forward.  Two Matrices need to have the same dimensions in order to be summed, and the sum is just the sum of the entries, so:

$$\begin{pmatrix} a & b \\ c & d \end{pmatrix}+ \begin{pmatrix} w & x \\ y & z \end{pmatrix}=\begin{pmatrix} a+w & b+x \\ c+y & d+z \end{pmatrix}$$


\begin{example}
Company $B$ gets in on this action and will give 2 dollars to each charity for each student who attends.  NOW given $F$ female and $M$ male students, how much money will go to Elderly and the Homeless?

\begin{eqnarray*}
\begin{pmatrix} 0.8 & 0.5 & 0 \\ 0.2 & 0.5 & 1 \end{pmatrix} \left( \begin{pmatrix} 2 & 1 \\ 1 & 2 \\ 2 & 2\end{pmatrix} + \begin{pmatrix} 2 & 2 \\ 2 & 2 \\ 2 & 2\end{pmatrix}   \right)\begin{pmatrix} F\\M \end{pmatrix}&=&\begin{pmatrix} 0.8 & 0.5 & 0 \\ 0.2 & 0.5 & 1 \end{pmatrix} \begin{pmatrix} 4 & 3 \\ 3 & 4 \\ 4 & 4\end{pmatrix} \begin{pmatrix} F\\M \end{pmatrix}\\
&=&\begin{pmatrix} 4.7 & 4.4 \\ 7.3 &  6.6 \end{pmatrix}\begin{pmatrix} F\\M \end{pmatrix}\\
&=&\begin{pmatrix} 4.7F+4.4M\\ 7.3F+6.6M \end{pmatrix}\\
\end{eqnarray*}
So \$$4.7 F+4.4M$ for Elderly and $\$7.3 F+6.6M$ for Homeless.
\end{example}

\section{Computation using Sage}

As usual, this seems like it should be much easier using technology.  Let's suppose $A=\begin{pmatrix} 1 & 2 \\ 3 & 4\end{pmatrix}, B=\begin{pmatrix} 5 & 6 \\ 7 & 8\end{pmatrix}$.  Verify that:

\begin{eqnarray*}
A+B&=&\begin{pmatrix} 6 & 8 \\ 10 & 12\end{pmatrix}\\
AB&=&\begin{pmatrix} 19 & 22 \\ 43 & 50\end{pmatrix}\\
BA&=&\begin{pmatrix} 23 & 34 \\ 31 & 46\end{pmatrix}
\end{eqnarray*}


\url{https://sagecell.sagemath.org/?z=eJxztM1NLCnKrNAIDNRRiI421DGK1Yk21jGJjdXk5XJClTTVMQNKmutYgCULijLzSjQctZ0QbC0E20nLURMAu6IZgA==&lang=sage}




\end{document}
