\documentclass[10pt]{article}

% The following command leaves more space between lines.  That's great
% when correcting drafts.  When you comment it out, however, the
% output looks much nicer.
%
\linespread{1.0}

\usepackage{amsmath}
\usepackage{amssymb}
\usepackage{graphicx}
\usepackage{epsfig}
\usepackage{latexsym}
\usepackage{amsthm}

\usepackage{mathrsfs}

\usepackage{multicol}



\usepackage[colorlinks,citecolor=blue]{hyperref}

\usepackage[latin1]{inputenc}

\usepackage{tikz-cd}
\usepackage{pgfplots}

%\usepackage{3dplot}

\usetikzlibrary{matrix,arrows,decorations.pathmorphing}


\usepackage[scale=0.8]{geometry}


%\usepackage{umoline}\setlength{\UnderlineDepth}{1pt}
%\usepackage[linktocpage=true]{hyperref}

\input xy
\xyoption{all}


%\addtolength{\hoffset}{-.5in}
%\addtolength{\textwidth}{1in}
%\setlength{\parindent}{.5in}
%\setlength{\textheight}{9.5in} \setlength{\topmargin}{-2cm}


\pagestyle{myheadings}\parindent 0em




\usepackage[latin1]{inputenc}


%------------------copy and posted code from the internets-------------

%\numberwithin{equation}{section} % comment out when neccessary

\newtheorem{theorem}[equation]{Theorem}
\newtheorem{lemma}[equation]{Lemma}
\newtheorem{proposition}[equation]{Proposition}
\newtheorem{corollary}[equation]{Corollary}


\theoremstyle{definition}
\newtheorem{definition}[equation]{Definition}
\newtheorem{example}[equation]{Example}
\newtheorem{remark}[equation]{Remark}
\newtheorem{problem}[equation]{Problem}



\newcommand{\R}[1]{\mathbb{R}^{#1}}
\newcommand{\C}[1]{\mathbb{C}^{#1}}
\newcommand{\Z}[1]{\mathbb{Z}^{#1}}
\newcommand{\K}[1]{\mathbb{K}^{#1}}
\newcommand{\embed}[0]{\hookrightarrow}
\newcommand{\TT}[4]{\begin{tabular}{| c | c |}\hline $#1$ & $#2$ \\ \hline $#3$ & $#4$ \\ \hline\end{tabular}} %goddamn it
\newcommand{\partd}[2]{\frac{\partial #1}{\partial #2}}
\newcommand{\limit}[2]{\displaystyle{ \lim_{#1 \to #2}}}
\newcommand{\vectornorm}[1]{\left|\left|#1\right|\right|}
\newcommand{\Ker}[0]{\text{\textnormal{Ker}}}
\newcommand{\Hom}[0]{\text{\textnormal{Hom}}}
\newcommand{\circled}[1]{\tikz[baseline=(char.base)]{
            \node[shape=circle,draw,inner sep=2pt] (char) {#1};}}


\newcommand{\T}{\rotatebox[origin=c]{180}{$\scriptscriptstyle \perp $}}
\newcommand{\x}{\textbf{x}}
\newcommand{\y}{\textbf{y}}
\newcommand{\supp}{\text{\textnormal{supp}}}
\newcommand{\csupp}{\text{\textnormal{cosupp}}}
\newcommand{\found}{\text{\textnormal{found}}}
\newcommand{\roof}{\text{\textnormal{roof}}}

\newcommand{\bcup}{\displaystyle\bigcup}
\newcommand{\bcap}{\displaystyle\bigcap}
\newcommand{\dsum}{\displaystyle\sum}
\newcommand{\dint}{\displaystyle\int}





\begin{document}
%

{\bf Name:} \hrulefill\hrulefill\hrulefill\\
{\bf M143} \qquad \qquad \\
{\bf Exponential  and Logarithmic Functions}\\ %(look familiar??)\\
%Show all work for full/partial credit.
%---------------- End of the document ---------------

\section{Exponential}

\begin{definition}
An \textbf{exponential function} is a function $f(x)$ which may be written as $f(x)=b\cdot a^x$.
\end{definition}

We should note that we've seen exponential functions before, in the guise of compound interest.  If the future value of a debt is $S=P(1+i)^x$ after $x$ unites of time, then $P$ is your $b$ and $1+i$ is your $a$.\\

We should also note that there is a real kinship between exponential functions and linear functions.  Linear functions have the form $\ell(x)=mx+b$, which we can think of as $\ell(x)=\overbrace{m+m+\cdots+m}^x+b$, with $x$ copies of $m$.  Similarly we can see that $f(x)=b\cdot a^x=\overbrace{a\cdot a\cdots a}^x\cdot b$, so it is really the multiplicative analogue of the linear function, where the $b$'s are the initial value of the function when $x=0$ and the growth rate of the function is totally determined by $a$ (whereas it's determined by $m$ for linear functions).\\

In that way, we can also tell when $f(x)=b\cdot a^x$ is increasing or decreasing, just by looking at whether or not $a>1$ or $a<1$.  If $a>1$ it is increasing: \url{https://www.desmos.com/calculator/edktzphc6x} if $a<1$ then it's decreasing: \url{https://www.desmos.com/calculator/rpiu9qc4ty}.\\


Exponential functions are used to model anything that grows or decays proportionately to  it's current value.  So for example debts or investment accrue proportionately to how much debt/investment there already is.  Population is another example, the greater the population, the more reproduction there will be within that population and the greater the increase in population.

\section{Logarithmic}

So looking at any exponential function, but specifically for increasing ones, we notice that they are 1-1, meaning two different inputs give you 2 different outputs.  Thus, $f(x)=a^x$ is an invertible function.  You can see in these graphs \url{https://www.desmos.com/calculator/cvriy608bg} that reflecting the exponential over the $y=x$ line giving us the green function is another function.  We call this function $\log_a(x)$.  As the inverse of an exponential function, it has some properties:

\begin{itemize}
\item \textbf{It's domain is only positive numbers}.  Why?  Because the only possible outputs of exponential outputs are positive numbers.  Taking the inverse reverses the roles of the domain and range, and so the domain of $\log_a$ is the positives.
\item \textbf{As $\mathbf{x}$ goes to 0, $ \mathbf{\log_a(x)}$ goes to $\mathbf{-\infty}$.}   This is the result of the reflection.  Normally $a^x$ (for $a>1$) goes to 0 as $x\to -\infty$, so when reflected, that line is now asymptotic to the $y$-axis.  Also $\log_a$ returns the power necessary to achieve the value $x$.  If $x$ is a small number like 0.0001, what power would I have to raise $a$ to to get $0.0001$?  It can't be 0, $a^0=1$, so it has to be \textbf{less} than 0, and the smaller $x$ is, the lower this power must go.

\item  As $\mathbf{x\to\infty, \log_a(x)\to \infty}$, again this is the result  of the reflection, but we can also think of this as the powers $a$ needs to be raised to in order to achieve $x$, as this increases, those powers must increase as well.

\item $\log_a(xy)=\log_a(x)+\log_a(y)$. Too see this consider:

\begin{eqnarray*}
xy&=&a^{\log_a{xy}}\\
xy&=&x\cdot y\\
&=&a^{\log_a(x)}\cdot a^{\log_a(y)}\\
&=&a^{\log_a(x)+\log_a(y)}.
\end{eqnarray*}


\item $\log_a(x^y)=y\log_a(x)$.  To see this, consider:

\begin{eqnarray*}
x^y&=&a^{\log_a(x^y)}.\\
x^y&=&(a^{\log_a(x)})^y\\
&=&a^{y\log_a(x)}
\end{eqnarray*}



\end{itemize}

Typically most textbooks include a whole bunch of other arithmetic rules, but they can all be distilled from the 2 above so they're all totally pointless.\\

Special cases of logs are log base 10 which is usually just denoted $\log$ and log based $e$, which is denoted $\ln$.  Astronomers and other scientists like $\log$ since taking $\log$ base 10 gives you approximate magnitude of stuff.  As a mathematician, I prefer $\ln$ since $e$ has pretty special mathematical properties, plus it's shorter, which makes it better.\\

The main use of logs algebraically is to de-exponentiate expressions.  So if one is trying to solve $10=2^x$, we could do this via:

\begin{eqnarray*}
10&=&2^x\\
\ln(10)&=&\ln(2^x)\\
\ln(10)&=&x\ln(2)\\
x&=&\frac{\ln(10)}{\ln(2)}\approx 3.3219.
\end{eqnarray*}

One can also solve problems like this visually:  \url{https://www.desmos.com/calculator/t0pojr0gel}.






























\end{document}
