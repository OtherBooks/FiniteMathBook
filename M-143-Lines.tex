\documentclass[10pt]{article}

% The following command leaves more space between lines.  That's great
% when correcting drafts.  When you comment it out, however, the
% output looks much nicer.
%
\linespread{1.0}

\usepackage{amsmath}
\usepackage{amssymb}
\usepackage{graphicx}
\usepackage{epsfig}
\usepackage{latexsym}
\usepackage{amsthm}

\usepackage{mathrsfs}

\usepackage{multicol}



\usepackage[colorlinks,citecolor=blue]{hyperref}

\usepackage[latin1]{inputenc}

\usepackage{tikz-cd}
\usepackage{pgfplots}

%\usepackage{3dplot}

\usetikzlibrary{matrix,arrows,decorations.pathmorphing}


\usepackage[scale=0.8]{geometry}


%\usepackage{umoline}\setlength{\UnderlineDepth}{1pt}
%\usepackage[linktocpage=true]{hyperref}

\input xy
\xyoption{all}


%\addtolength{\hoffset}{-.5in}
%\addtolength{\textwidth}{1in}
%\setlength{\parindent}{.5in}
%\setlength{\textheight}{9.5in} \setlength{\topmargin}{-2cm}


\pagestyle{myheadings}\parindent 0em




\usepackage[latin1]{inputenc}


%------------------copy and posted code from the internets-------------

%\numberwithin{equation}{section} % comment out when neccessary

\newtheorem{theorem}[equation]{Theorem}
\newtheorem{lemma}[equation]{Lemma}
\newtheorem{proposition}[equation]{Proposition}
\newtheorem{corollary}[equation]{Corollary}


\theoremstyle{definition}
\newtheorem{definition}[equation]{Definition}
\newtheorem{example}[equation]{Example}
\newtheorem{remark}[equation]{Remark}
\newtheorem{problem}[equation]{Problem}



\newcommand{\R}[1]{\mathbb{R}^{#1}}
\newcommand{\C}[1]{\mathbb{C}^{#1}}
\newcommand{\Z}[1]{\mathbb{Z}^{#1}}
\newcommand{\K}[1]{\mathbb{K}^{#1}}
\newcommand{\embed}[0]{\hookrightarrow}
\newcommand{\TT}[4]{\begin{tabular}{| c | c |}\hline $#1$ & $#2$ \\ \hline $#3$ & $#4$ \\ \hline\end{tabular}} %goddamn it
\newcommand{\partd}[2]{\frac{\partial #1}{\partial #2}}
\newcommand{\limit}[2]{\displaystyle{ \lim_{#1 \to #2}}}
\newcommand{\vectornorm}[1]{\left|\left|#1\right|\right|}
\newcommand{\Ker}[0]{\text{\textnormal{Ker}}}
\newcommand{\Hom}[0]{\text{\textnormal{Hom}}}
\newcommand{\circled}[1]{\tikz[baseline=(char.base)]{
            \node[shape=circle,draw,inner sep=2pt] (char) {#1};}}


\newcommand{\T}{\rotatebox[origin=c]{180}{$\scriptscriptstyle \perp $}}
\newcommand{\x}{\textbf{x}}
\newcommand{\y}{\textbf{y}}
\newcommand{\supp}{\text{\textnormal{supp}}}
\newcommand{\csupp}{\text{\textnormal{cosupp}}}
\newcommand{\found}{\text{\textnormal{found}}}
\newcommand{\roof}{\text{\textnormal{roof}}}

\newcommand{\bcup}{\displaystyle\bigcup}
\newcommand{\bcap}{\displaystyle\bigcap}
\newcommand{\dsum}{\displaystyle\sum}
\newcommand{\dint}{\displaystyle\int}





\begin{document}
%

{\bf Name:} \hrulefill\hrulefill\hrulefill\\
{\bf M143} \qquad \qquad \\
{\bf Lines}\\ %(look familiar??)\\
%Show all work for full/partial credit.
%---------------- End of the document ---------------



\section{Lines in the plane}


I imagine that the algebraic definition of a line in the cartesian plane is something that we may all be familiar with.  But I'm going to approach this section  conceptually first and we will see that the algebraic formulations follow naturally.  This is a theme in mathematics that is under, or never, emphasized: \textbf{CONCEPTS COME FIRST, ALGEBRA IS A CONSEQUENCE.}. In short: Algebra is an afterthought to ideas.\\

So imagine your an ant standing on an infinite table that extends in all directions, and someone tells you to walk in a line.  Well, naturally there are infinitely many possible lines along which you could walk, so you could respond with ``Hey, which line do you want me to walk along?".  Supposing that the person giving you instructions is somewhat lazy, they'd like to communicate this to you as concisely as possible.\\

So how could someone communicate an exact line to you?\\

Well, they could specify a point, they could say 	``Hey, we want you to start exactly where you're standing".  That's great, but looking around you, there are certainly many, in fact infinite, lines along which you could traverse.  So at this point, a bit more information is needed.


\subsection{Point \& Slope}


So one but of more information that could be applied, is a direction.  They could point in a direction and say ``Hey, start where you are and walk THAT way."  That certainly would describe a line.  What this translates to algebraically is a {\bf point} and a {\bf slope}.   The slope measures how many units of $y$ your line increases (or decreases by) per 1 unit increase of $x$.  Changing this value essentially changes the trajectory of your line.\\

So, typically lines are defined in the form $y=mx+b$, where $m$ is the slope and $b$ is the $y$-intercept.  The notion of a $y$-intercept just means that when $x=0$, it follows that the $y$ value is $y=m(0)+b=b$, the the point on the $y$-axis will be $(0,b)$.  It can be convenient to also define a line in the form $$y-y_0=m(x-x_0),$$ where $(x_0, y_0)$ is a point that falls on your line.\\

Why would we care about a second formulation, and how do we know it's even a line?  To answer the first question, we recall that ANY geometric object defined by an equation is the set of points that make the equation true.  So given the equation $y-y_0=m(x-x_0)$, if we plug in $(x_0, y_0)$, we would get $y_0-y_0=m(x_0-x_0)$ or $0=0$ which is definitely a true statement.  So whatever shape we get from $y-y_0=m(x-x_0)$, it contains $(x_0, y_0)$.\\

We know it's a line because:

\begin{eqnarray*}
y-y_0&=&m(x-x_0)\\
y&=&mx-mx_0+y_0\\
y&=&mx+(y_0-mx_0)
\end{eqnarray*}

so if we let $b=y_0-mx_0$, we get the standard form for a line.


\begin{example}
Find the line with slope 3 passing through $(2,5)$.\\

One way we could do this is to note that since we are dealing with lines, the equation we are trying to obtain has the form $y=mx+b$.  We know that $m=3$, and that $x=2, y=5$ satisfies this equation.  So:

\begin{eqnarray*}
y&=&mx+b\\
5&=&3(2)+b\\
b&=&5-6=-1\\
y&=&3x-1.
\end{eqnarray*}

On the other hand:  Knowing that $(2,5)$ lies on the line, we can let $x_0=2, y_0=5$ and thus:

\begin{eqnarray*}
y-y_0&=&m(x-x_0)\\
y-5&=&3(x-2)\\
y&=&3x-6+5\\
y&=&3x-1.
\end{eqnarray*}

Either way, you obtain the same line, with slope 3, passing through (2,5): \url{https://www.desmos.com/calculator/bc8wwmozsp}.

\end{example}

\subsection{Two points}

On the other hand, instead of giving you a direction, one could be given another point the traverse through.  If you imagine any two dots on an infinite table, it's not hard to see you can draw a line between them.  The exact way we do this also follows intuitively.  If you were at home, and you had to go to the store, the natural steps you would take to do so are:

\begin{enumerate}
\item Figure out which way to the store.
\item Go there.
\end{enumerate}

So we have to figure out the direction between these points, which in our analogy means the slope between the points.  Remember the slope measures how much $y$ changes per change in $x$.  So, the measurement of slope follows from the change in $y$ over the change in $x$.  So to measure this, if we have points $(x_0, y_0), (x_1, y_1)$, we want to see how much $y$ changes over how much $x$ changes, or:

$$m=\frac{y_1-y_0}{x_1-x_0}$$
or equivalently $m=\frac{y_0-y_1}{x_0-x_1}$.


\begin{example}
What line passes between $(2,3)$ and $(4,1)$?\\

So we note then that $m=\frac{1-3}{4-2}=-1$ (also $m=\frac{3-1}{2-4}=-1$).  Once the slope is identified, we can use any point, and any method to find the line.  I want go through all the possibilities, but consider:

\begin{eqnarray*}
y&=&mx+b\\
3&=&(-1)(2)+b\\
b&=&3+2=5\\
y&=&-x+5.
\end{eqnarray*}

Or:

\begin{eqnarray*}
y-y_0&=&m(x-x_0)\\
y-1&=&(-1)(x-4)\\
y&=&-x+4+1\\
y&=&-x+5.
\end{eqnarray*}

Either way, we identify the same line: \url{https://www.desmos.com/calculator/yohrdbuwuc}

\end{example}

\section{So I have a line.  Now what?}

Once you obtain a line, this gives you a relationship between the $x$ and $y$ variables.  It allows you identify, given an $x$ or $y$ value, the other variable:

\begin{example}
Consider the line $y=2x-3$.
\begin{enumerate}
\item When $y=0$, what is $x$?
\item When $x=3$ what is $y$?

These can be identified algebraically:
\begin{enumerate}
\item When $y=0$, we have:
\begin{eqnarray*}
y&=&2x-3\\
0&=&2x-3\\
3&=&2x\\
x&=&1.5
\end{eqnarray*}
\item if $x=3$:
\begin{eqnarray*}
y&=&2x-3\\
y&=&2(3)-3\\
y&=&6-3=3.
\end{eqnarray*}
We can identify both points graphically as well: \url{https://www.desmos.com/calculator/gukgr4sjfm}.
\end{enumerate}


\end{enumerate}

\end{example}














\end{document}
