\documentclass[10pt]{article}

% The following command leaves more space between lines.  That's great
% when correcting drafts.  When you comment it out, however, the
% output looks much nicer.
%
\linespread{1.0}

\usepackage{amsmath}
\usepackage{amssymb}
\usepackage{graphicx}
\usepackage{epsfig}
\usepackage{latexsym}
\usepackage{amsthm}

\usepackage{mathrsfs}

\usepackage{multicol}



\usepackage[colorlinks,citecolor=blue]{hyperref}

\usepackage[latin1]{inputenc}

\usepackage{tikz-cd}
\usepackage{pgfplots}

%\usepackage{3dplot}

\usetikzlibrary{matrix,arrows,decorations.pathmorphing}


\usepackage[scale=0.8]{geometry}


%\usepackage{umoline}\setlength{\UnderlineDepth}{1pt}
%\usepackage[linktocpage=true]{hyperref}

\input xy
\xyoption{all}


%\addtolength{\hoffset}{-.5in}
%\addtolength{\textwidth}{1in}
%\setlength{\parindent}{.5in}
%\setlength{\textheight}{9.5in} \setlength{\topmargin}{-2cm}


\pagestyle{myheadings}\parindent 0em




\usepackage[latin1]{inputenc}


%------------------copy and posted code from the internets-------------

%\numberwithin{equation}{section} % comment out when neccessary

\newtheorem{theorem}[equation]{Theorem}
\newtheorem{lemma}[equation]{Lemma}
\newtheorem{proposition}[equation]{Proposition}
\newtheorem{corollary}[equation]{Corollary}


\theoremstyle{definition}
\newtheorem{definition}[equation]{Definition}
\newtheorem{example}[equation]{Example}
\newtheorem{remark}[equation]{Remark}
\newtheorem{problem}[equation]{Problem}



\newcommand{\R}[1]{\mathbb{R}^{#1}}
\newcommand{\C}[1]{\mathbb{C}^{#1}}
\newcommand{\Z}[1]{\mathbb{Z}^{#1}}
\newcommand{\K}[1]{\mathbb{K}^{#1}}
\newcommand{\embed}[0]{\hookrightarrow}
\newcommand{\TT}[4]{\begin{tabular}{| c | c |}\hline $#1$ & $#2$ \\ \hline $#3$ & $#4$ \\ \hline\end{tabular}} %goddamn it
\newcommand{\partd}[2]{\frac{\partial #1}{\partial #2}}
\newcommand{\limit}[2]{\displaystyle{ \lim_{#1 \to #2}}}
\newcommand{\vectornorm}[1]{\left|\left|#1\right|\right|}
\newcommand{\Ker}[0]{\text{\textnormal{Ker}}}
\newcommand{\Hom}[0]{\text{\textnormal{Hom}}}
\newcommand{\circled}[1]{\tikz[baseline=(char.base)]{
            \node[shape=circle,draw,inner sep=2pt] (char) {#1};}}


\newcommand{\T}{\rotatebox[origin=c]{180}{$\scriptscriptstyle \perp $}}
\newcommand{\x}{\textbf{x}}
\newcommand{\y}{\textbf{y}}
\newcommand{\supp}{\text{\textnormal{supp}}}
\newcommand{\csupp}{\text{\textnormal{cosupp}}}
\newcommand{\found}{\text{\textnormal{found}}}
\newcommand{\roof}{\text{\textnormal{roof}}}

\newcommand{\bcup}{\displaystyle\bigcup}
\newcommand{\bcap}{\displaystyle\bigcap}
\newcommand{\dsum}{\displaystyle\sum}
\newcommand{\dint}{\displaystyle\int}





\begin{document}
%

{\bf Name:} \hrulefill\hrulefill\hrulefill\\
{\bf M143} \qquad \qquad \\
{\bf Rational Functions}\\ %(look familiar??)\\
%Show all work for full/partial credit.
%---------------- End of the document ---------------

\begin{definition}
A \textbf{rational function} is a function $r(x)$ which may be written as $r(x)=\frac{p(x)}{q(x)}$ where $p(x), q(x)$ are both polynomials, and $q(x)\neq 0$.
\end{definition}

Now when we say $q(x)\neq0$, we mean that $q(x)$ is not the polynomial 0, not that $q(x)$ could never be 0.  As before, we will consider the long and short term behavior of polynomials.

\section{Long Term Behavior}

For Polynomials, what determined the long term behavior of a polynomial was the leading term.  So it makes sense that a rational function will be determined the same way.  Given  $$r(x)=\frac{a_nx^n+\cdots a_0x^0}{b_mx^m+\cdots b_0x^0},$$ we can determine the long term behavior with $$\frac{a_nx^n}{b_mx^m}$$ which simplifies to $\frac{a_n}{b_m}x^{n-m}$.  So there are some cases to consider here:

\begin{itemize}
\item If $n>m$, then $\frac{a_n}{b_m}x^{n-m}$ is just a regular monomial.  This doesn't mean that $r(x)$'s graph will look like a polynomial's, but it does mean the long term behavior will be the same as that of a polynomial.\\

For example consider $y=\frac{-x^4+x}{x+1}$.  We should note: $\frac{-x^4-x}{x+1}\sim\frac{-x^4}{x}
\sim -x^3$, so the long term behavior of $y$ will be the same as $-x^3$'s:  positive infinity to the left, negative infinity to the right.  \url{https://www.desmos.com/calculator/jzxyyrcgep}.

\item If $n=m$, then $\frac{a_n}{b_m}x^{n-m}=\frac{a_n}{b_m}$ is just a constant.  So as the magnitude of $x$ grows, these values approach $\frac{a_n}{b_m}$.\\

For example consider $y=\frac{-2x^4+x}{3x^4+1}$.  We should note: $\frac{-2x^4-x}{3x^4+1}\sim\frac{-2x^4}{3x^4}
\sim -\frac{2}{3}$, so the long term behavior of $y$ will be the same as $-\frac{2}{3}$'s:   \url{https://www.desmos.com/calculator/qaeq86mivp}.
 
\item If $n<m$, then $\frac{a_n}{b_m}x^{n-m}=\frac{a_n}{b_m}\frac{1}{x^{m-n}}$.  As the magnitude of $x$ grows, the more $\frac{1}{x^{m-n}}$ will shrink and the entire function converges to 0.\\

For example consider $y=\frac{-2x^2+x}{3x^4+1}$.  We should note: $\frac{-2x^2-x}{3x^4+1}\sim\frac{-2x^2}{3x^4}
\sim -\frac{2}{3}\frac{1}{x^2}$, so the long term behavior of $y$ will be converging to 0:   \url{https://www.desmos.com/calculator/n0ftbxmcod}.
 
 
\end{itemize}


\section{Short term behavior}

Here, we once again care about the multiplicity of the roots, that is, if $r(x)=\frac{p(x)}{q(x)}$, what we care about is when $p(x)$ or $q(x)$ is 0, and what the multiplicity is.\\

Well, turns out, 0 divided by anything is 0.  So if the numerator is 0, then $r(x)$ is 0.  Moreover, it bounces or crosses in the same way as a polynomial.  So if the multiplicity is odd, the function crosses and if it's even, it bounces.\\

For example, consider $r_1(x)=\frac{x-1}{x}$.  It is clearly 0 when $x=1$, and since the multiplicity is odd, we cross \url{https://www.desmos.com/calculator/9graa40kq1}.  But if we change that to $r_2(x)=\frac{(x-1)^2}{x}$, the multiplicity is now even so we bounce \url{https://www.desmos.com/calculator/iy9kvoac37}.\\

It also turns out you can't divide by 0.  So at those points, the function is undefined.  And as the values in the denominator towards 0, the overall value of the quotient approaches either positive or negative infinity.  Just like roots, whether or not we change signs at these asymptotes depends on the multiplicity of the root.\\




For example, again consider $r_1(x)=\frac{x-1}{x}$.  The denominator is 0 when $x=0$, so we have a vertical asymptote there.  Since the multiplicity is odd, we change signs there: \url{https://www.desmos.com/calculator/ik9xwhui5i}.  If we made that multiplicity even, the function would not change sign at $x=0$, so for $r_2(x)=\frac{x-1}{x^2}$: \url{https://www.desmos.com/calculator/h5famtu30t}



































\end{document}
