\documentclass[10pt]{article}

% The following command leaves more space between lines.  That's great
% when correcting drafts.  When you comment it out, however, the
% output looks much nicer.
%
\linespread{1.0}

\usepackage{amsmath}
\usepackage{amssymb}
\usepackage{graphicx}
\usepackage{epsfig}
\usepackage{latexsym}
\usepackage{amsthm}

\usepackage{mathrsfs}

\usepackage{multicol}



\usepackage[colorlinks,citecolor=blue]{hyperref}

\usepackage[latin1]{inputenc}

\usepackage{tikz-cd}
\usepackage{pgfplots}

%\usepackage{3dplot}

\usetikzlibrary{matrix,arrows,decorations.pathmorphing}


\usepackage[scale=0.8]{geometry}


%\usepackage{umoline}\setlength{\UnderlineDepth}{1pt}
%\usepackage[linktocpage=true]{hyperref}

\input xy
\xyoption{all}


%\addtolength{\hoffset}{-.5in}
%\addtolength{\textwidth}{1in}
%\setlength{\parindent}{.5in}
%\setlength{\textheight}{9.5in} \setlength{\topmargin}{-2cm}


\pagestyle{myheadings}\parindent 0em




\usepackage[latin1]{inputenc}


%------------------copy and posted code from the internets-------------

%\numberwithin{equation}{section} % comment out when neccessary

\newtheorem{theorem}[equation]{Theorem}
\newtheorem{lemma}[equation]{Lemma}
\newtheorem{proposition}[equation]{Proposition}
\newtheorem{corollary}[equation]{Corollary}


\theoremstyle{definition}
\newtheorem{definition}[equation]{Definition}
\newtheorem{example}[equation]{Example}
\newtheorem{remark}[equation]{Remark}
\newtheorem{problem}[equation]{Problem}



\newcommand{\R}[1]{\mathbb{R}^{#1}}
\newcommand{\C}[1]{\mathbb{C}^{#1}}
\newcommand{\Z}[1]{\mathbb{Z}^{#1}}
\newcommand{\K}[1]{\mathbb{K}^{#1}}
\newcommand{\embed}[0]{\hookrightarrow}
\newcommand{\TT}[4]{\begin{tabular}{| c | c |}\hline $#1$ & $#2$ \\ \hline $#3$ & $#4$ \\ \hline\end{tabular}} %goddamn it
\newcommand{\partd}[2]{\frac{\partial #1}{\partial #2}}
\newcommand{\limit}[2]{\displaystyle{ \lim_{#1 \to #2}}}
\newcommand{\vectornorm}[1]{\left|\left|#1\right|\right|}
\newcommand{\Ker}[0]{\text{\textnormal{Ker}}}
\newcommand{\Hom}[0]{\text{\textnormal{Hom}}}
\newcommand{\circled}[1]{\tikz[baseline=(char.base)]{
            \node[shape=circle,draw,inner sep=2pt] (char) {#1};}}


\newcommand{\T}{\rotatebox[origin=c]{180}{$\scriptscriptstyle \perp $}}
\newcommand{\x}{\textbf{x}}
\newcommand{\y}{\textbf{y}}
\newcommand{\supp}{\text{\textnormal{supp}}}
\newcommand{\csupp}{\text{\textnormal{cosupp}}}
\newcommand{\found}{\text{\textnormal{found}}}
\newcommand{\roof}{\text{\textnormal{roof}}}

\newcommand{\bcup}{\displaystyle\bigcup}
\newcommand{\bcap}{\displaystyle\bigcap}
\newcommand{\dsum}{\displaystyle\sum}
\newcommand{\dint}{\displaystyle\int}





\begin{document}
%

{\bf Name:} \hrulefill\hrulefill\hrulefill\\
{\bf M143} \qquad \qquad \\
{\bf Future \& Past Value}\\ %(look familiar??)\\
%Show all work for full/partial credit.
%---------------- End of the document ---------------

\section{Future Value}

Let's say you have a back account that gives you 2\% monthly interest (lucky you!)  and let's also say that from January to May, you deposit \$100 each month.  How much will you have in May? Well:

\begin{itemize}
\item The 100 dollars you put in January would accrue interests for 4 months, so you'd have $100(1.02)^4$ dollars.
\item The 100 dollars you put in February would accrue interests for 3 months, so you'd have $100(1.02)^3$ dollars.
\item The 100 dollars you put in March would accrue interests for 2 months, so you'd have $100(1.02)^2$ dollars.
\item The 100 dollars you put in April would accrue interests for 1 months, so you'd have $100(1.02)^1$ dollars.
\item The 100 dollars you put in May would accrue interests for 0 months, so you'd have $100(1.02)^0$ dollars.
\end{itemize}

So you'd have: $$100(1.02)^4+100(1.02)^3+100(1.02)^2+100(1.02)^1+100(1.02)^0\approx \$520.40$$

But what if this went on for 10 months?  For 10 years?  You probably wouldn't want to calculate every single one of these.  So is there an easier way to compute something like $$100(1.02)^4+100(1.02)^3+100(1.02)^2+100(1.02)^1+100(1.02)^0?$$

So we have a sum of things that look like: $x^n+x^{n-1}+\cdots+x^1+x^0$.  There is a funny fact about this sum.  if we multiply it by $(x-1)$:

\begin{eqnarray*}
(x^n+x^{n-1}+\cdots+x^1+x^0)(x-1)&=&(x^n+x^{n-1}+\cdots+x^1+x^0)x-(x^n+x^{n-1}+\cdots+x^1+x^0)\\
&=&x^{n+1}+x^n+\cdots+x-x^n-x^{n-1}-\cdots-x^0\\
&=&x^{n+1}-x^0\\
&=&x^{n+1}-1
\end{eqnarray*}

Thus:

\begin{eqnarray*}
(x^n+x^{n-1}+\cdots+1)(x-1)&=&x^{n+1}-1\\
x^n+x^{n-1}+\cdots+1&=&\frac{x^{n+1}-1}{x-1}
\end{eqnarray*}


So, we should note that the future value of our investment above is:

\begin{eqnarray*}
100(1.02)^4+100(1.02)^3+100(1.02)^2+100(1.02)^1+100(1.02)^0&=&100((1.02)^4+(1.02)^3+(1.02)^2+(1.02)^1+(1.02)^0)\\
&=&100\frac{(1.02)^5-1}{1.02-1}\\
S&=&100\frac{(1.02)^5-1}{.02}
\end{eqnarray*}

Where $S$ is the future value.  Now, if we did this for $n$ months instead of 5, we would have:

$$S=100\frac{(1.02)^n-1}{.02}$$

If the monthly (or whatever time unit) interest rate was $i$ instead of 2\%, we would get:

$$S=100\frac{(1+i)^n-1}{i}$$

And if the payments were $R$ instead of \$100, we would have:

$$S=R\frac{(1+i)^n-1}{i}$$


So here we have the formula for the  future value of a debt or investment based on a constant payment.

\section{Present Value}

So if we have $n$ payments of $R$ at regular interest rate $i$, the future value is $S=R\frac{(1+i)^n-1}{i}$.  What is it's present value?  Well remember that the relationship between present and future value is:

\begin{eqnarray*}
S&=&P(1+i)^n\\
P&=&\frac{S}{(1+i)^n}.
\end{eqnarray*}
From this we get:

\begin{eqnarray*}
P&=&\frac{R\frac{(1+i)^n-1}{i}}{(1+i)^n}\\
&=&R\frac{((1+i)^n-1)\frac{1}{(1+i)^n}}{i}\\
&=&R\frac{1-\frac{1}{(1+i)^n}}{i}\\
P&=&R\frac{1-(1+i)^{-n}}{i}
\end{eqnarray*}

Suppose that we invest \$100,000 in an account that gives 1\% monthly interest and I'd like to pull money out for the next 10 years.  What is the Max that I could pull out?\\

Well, we have a present value of $P=100,000$, a monthly interest rate of $i=0.01$ and $n=120$ months.  So:

\begin{eqnarray*}
100000&=&R\frac{1-(1.01)^{-120}}{.01}\\
1000&\approx&R(1-0.302995)\\
R&\approx&\frac{1000}{0.697005}\\
&\approx&\$1434.71
\end{eqnarray*}

So with an investment like that, you can pull out \$1434.71 for the next 10 years.



\end{document}
