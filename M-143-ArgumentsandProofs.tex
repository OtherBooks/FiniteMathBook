\documentclass[10pt]{article}

% The following command leaves more space between lines.  That's great
% when correcting drafts.  When you comment it out, however, the
% output looks much nicer.
%
\linespread{1.0}

\usepackage{amsmath}
\usepackage{amssymb}
\usepackage{graphicx}
\usepackage{epsfig}
\usepackage{latexsym}
\usepackage{amsthm}

\usepackage{mathrsfs}

\usepackage{multicol}



\usepackage[colorlinks,citecolor=blue]{hyperref}

\usepackage[latin1]{inputenc}

\usepackage{tikz-cd}
\usepackage{pgfplots}

%\usepackage{3dplot}

\usetikzlibrary{matrix,arrows,decorations.pathmorphing}
\usepackage{circuitikz}

\usepackage[scale=0.8]{geometry}


%\usepackage{umoline}\setlength{\UnderlineDepth}{1pt}
%\usepackage[linktocpage=true]{hyperref}

\input xy
\xyoption{all}


%\addtolength{\hoffset}{-.5in}
%\addtolength{\textwidth}{1in}
%\setlength{\parindent}{.5in}
%\setlength{\textheight}{9.5in} \setlength{\topmargin}{-2cm}


\pagestyle{myheadings}\parindent 0em




\usepackage[latin1]{inputenc}


%------------------copy and posted code from the internets-------------

%\numberwithin{equation}{section} % comment out when neccessary

\newtheorem{theorem}[equation]{Theorem}
\newtheorem{lemma}[equation]{Lemma}
\newtheorem{proposition}[equation]{Proposition}
\newtheorem{corollary}[equation]{Corollary}


\theoremstyle{definition}
\newtheorem{definition}[equation]{Definition}
\newtheorem{example}[equation]{Example}
\newtheorem{remark}[equation]{Remark}
\newtheorem{problem}[equation]{Problem}



\newcommand{\R}[1]{\mathbb{R}^{#1}}
\newcommand{\C}[1]{\mathbb{C}^{#1}}
\newcommand{\Z}[1]{\mathbb{Z}^{#1}}
\newcommand{\K}[1]{\mathbb{K}^{#1}}
\newcommand{\embed}[0]{\hookrightarrow}
\newcommand{\TT}[4]{\begin{tabular}{| c | c |}\hline $#1$ & $#2$ \\ \hline $#3$ & $#4$ \\ \hline\end{tabular}} %goddamn it
\newcommand{\partd}[2]{\frac{\partial #1}{\partial #2}}
\newcommand{\limit}[2]{\displaystyle{ \lim_{#1 \to #2}}}
\newcommand{\vectornorm}[1]{\left|\left|#1\right|\right|}
\newcommand{\Ker}[0]{\text{\textnormal{Ker}}}
\newcommand{\Hom}[0]{\text{\textnormal{Hom}}}
\newcommand{\circled}[1]{\tikz[baseline=(char.base)]{
            \node[shape=circle,draw,inner sep=2pt] (char) {#1};}}


\newcommand{\T}{\rotatebox[origin=c]{180}{$\scriptscriptstyle \perp $}}
\newcommand{\x}{\textbf{x}}
\newcommand{\y}{\textbf{y}}
\newcommand{\supp}{\text{\textnormal{supp}}}
\newcommand{\csupp}{\text{\textnormal{cosupp}}}
\newcommand{\found}{\text{\textnormal{found}}}
\newcommand{\roof}{\text{\textnormal{roof}}}

\newcommand{\bcup}{\displaystyle\bigcup}
\newcommand{\bcap}{\displaystyle\bigcap}
\newcommand{\dsum}{\displaystyle\sum}
\newcommand{\dint}{\displaystyle\int}





\begin{document}
%

{\bf Name:} \hrulefill\hrulefill\hrulefill\\
{\bf M143} \qquad \qquad \\
{\bf Arguments \& Proofs}\\ %(look familiar??)\\
%Show all work for full/partial credit.
%---------------- End of the document ---------------



\section{Argument Types}

Here are some valid types of arguments.

\subsection{Modus Ponens}

$$\begin{array}{l}
p\to q \\
p\\
\hline
q
\end{array}$$

In other words, if you know $p\to q$ and $p$, then $q$ must be true.  So say ``If your pet is a pitbull, then it is a dog."  ``Your pet is a pitbull."  It would then be fair to conclude ``Therefore your pet is a dog."




\subsection{Modus Tollens}

$$\begin{array}{l}
p\to q \\
\sim q\\
\hline
\sim p
\end{array}$$

In other words, if you know $p\to q$ and $\sim q$, then $\sim p$ must be true.  So say ``If your pet is a pitbull, then it is a dog."  ``Your pet is not a dog."  It would then be fair to conclude ``Therefore your pet is not a pitbull."


\subsection{Disjunctive Syllogism}



$$\begin{array}{l}
p\vee q \\
\sim q\\
\hline
p
\end{array}$$
$$\begin{array}{l}
p\vee q \\
\sim p\\
\hline
q
\end{array}$$


If you know $p \vee q$ and one isn't true, then the other must be true.  ``I have to take either math or science next semester, I'm not going to take math, so therefore, I must take science."

\subsection{Reasoning by Transitivity}

$$\begin{array}{l}
p\to q \\
q\to r\\
\hline
p\to r
\end{array}$$

``If it rains then I will get wet.  If I get wet, I will get sick."  $\to $ ``Therefore, if it rains, I will get sick."


\section{Fallacies}

The following are common {\bf Fallacies}.  They seem like they could be true but they are not.

\subsection{Fallacy of the Converse}

$$\begin{array}{l}
p\to q \\
q\\
\hline
p
\end{array}$$



\subsection{Fallacy of the Inverse}

$$\begin{array}{l}
p\to q \\
\sim p\\
\hline
\sim q
\end{array}$$

The reason these are fallacies are covered in the previous write-up.


\section{Valid and invalid arguments}

So are the following valid?

\begin{example}
If I read the handouts and I watch the videos  then I will pass.  I didn't pass. I read the handouts, therefore I didn't watch the videos.\\

So let:

\begin{itemize}
\item $h=$``read handouts."
\item $v=$``watch videos."
\item $p=$``pass class."
\end{itemize}

So what we know is $(h\wedge v)\to p, h, \sim p.$  So:

$$\begin{array}{lcl}
(h\wedge v)\to p&\ \ \ &\\
\sim p\\
\hline
\sim(h\wedge v)=\sim h \vee\sim v & &\text{Modus Tollens}\\
 \\
\sim h \vee \sim v \\
h=\sim \sim h\\
\hline
\sim v  & &\text{Disjunctive Syllogism}
\end{array}$$

Therefore $\sim v$, I did not watch the videos.  This was a valid argument.


\end{example}


\begin{example}
If I drink coffee, I will be happy.  If I drink coffee, I will be wired.  If I am happy, then I am wired.\\

So let:

\begin{itemize}
\item $c=$``drink coffee."
\item $h=$``be happy."
\item $w=$``be wired."
\end{itemize}

What we have is $c\to h, c\to w$.  But there's nothing here to connect $h$ to $w$.  If we knew this person drank coffee, we could make that conclusion, but since we don't know that, if we knew that they were happy, it could be for some reason other than drinking coffee.  So we have no idea whether or not they'd be wired as well.  So we cannot conclude $h \to w$ and this is NOT a valid argument.


\end{example}












\end{document}
