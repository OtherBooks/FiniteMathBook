
\chapter{Financial Mathematics}\label{Chapter:Financial}


\section{Simple Interest}\label{Section:SimpleInterest}

Let's say you take out a loan at \$100 dollars and are charged 5\% monthly simple interest.  What this means is that every month, 5\% of your initial debt is added on as interest.  Over time you would see something like:

$$\begin{array}{|c|c|}
\hline
\text{Month} & \text{Debt}\\
\hline
0 & 100\\
1& 100+(.05)(100)=100+5=105\\
2& 105+(.05)(100)=105+5=110\\
3 & 110+(.05)(100)=110+5=115\\
\vdots & \vdots\\
n & 100+(.05)(100)n=100+5n\\
\hline
\end{array}$$

So in say 5 months you would owe $100+5*5=\$125$.  In general, we could track how much we owe $A$ in month $m$ as $A=100+5m$, which is, as we know, a linear equation: \url{https://www.desmos.com/calculator/m0k43cprhq} \\

Now of course, this could be adjusted to any number of situations.  You could take out a loan for $P$ dollars.  You could have a $r$ interest rate every time period which doesn't have to be a month.  Then the future value $A$ after $m$ time periods would be:

$$\begin{array}{|c|c|}
\hline
\text{Time Unit} & \text{Debt}\\
\hline
0 & P\\
1& P+rP\\
2& P+rP+rP=P+2rP\\
3 & P+2rP+rP=P+3rp\\
\vdots & \vdots\\
m & P+mrP=P(1+mr)\\
\hline
\end{array}$$

This still expresses $A$ as a line with $y$-intercept $P$ and slope $rP$: \url{https://www.desmos.com/calculator/hdj0ublnkk}\\

So suppose you lend your friend some money at 2\% weekly interest, and after 10 weeks they owe you \$240, how much did you lend them?  Well, we know that in this case $r=0.02, m=10$ and the future value $A=240$ so:

\begin{eqnarray*}
A&=&P+mrP\\
240&=&P+(10)(.02)P\\
240&=&1.2P\\
P&=&200
\end{eqnarray*}
 So $P=\$200$ is how much they borrowed.  One can verify this graphically: \url{https://www.desmos.com/calculator/lq0r43ctb0}.
 
 
 \section{Compound Interest}\label{Section:CompoundInterest}

Almost no meaningful financial transaction in the world is computed by simple interest.  The most common type of interest is compound interest which charges interest not just on your principle loan, but also on previously accrued interest as well.\\

So lets say you took out \$100 in a loan that charged 5\% monthly interest compounded monthly.  That means each month, you would owe 5\% more than what you owed the month before.


$$\begin{array}{|c|c|}
\hline
\text{Month} & \text{Debt}\\
\hline
0 & 100\\
1& 100+(.05)(100)=100+5=105\\
2& 105+(.05)(105)=105+5.25=110.25\\
3 & 110.25+(.05)(110.25)=110.25+5.5125=115.405\\
\vdots & \vdots\\
\hline
\end{array}$$

So not only is the debt increasing, but the amount that the debt increases per month, also increases.  Is there an easy way to see how much we'd owe in say 10 months without computing every month in between?\\

Well, If I owe $X$ dollars in any given month, then the next month I'll owe $X+0.05X=(1.05)X$ dollars.  In another month I'll owe $1.05X+(.05)(1.05)X=(1.05)(1.05)X=X(1.05)^2$ dollars.  If we apply this logic to our situation, we can see:


$$\begin{array}{|c|c|}
\hline
\text{Month} & \text{Debt}\\
\hline
0 & 100\\
1& 100+(.05)(100)=(1.05)100\\
2& (1.05)100+(.05)((1.05)100)=(1.05)(1.05)100=(1.05)^2 100\\
3 & (1.05)^2100+(.05)((1.05)^2100)=(1.05)(1.05)^2 100=(1.05)^3 100\\
\vdots & \vdots\\
n & (1.05)^n100\\
\hline
\end{array}$$

So in 10 months, we'd owe $100*(1.05)^10\approx \$162.89$.  In general, this forms an \textbf{exponential function}: \url{https://www.desmos.com/calculator/r51afj5zwr}


In general, if the interest rate per time unit is $i$, then after $n$ time units:


$$\begin{array}{|c|c|}
\hline
\text{Time Unit} & \text{Debt}\\
\hline
0 & P\\
1& P+iP=(1+i)P\\
2& (1+i)P+i(1+i)P=(1+i)^2P\\
3 & (1+i)^2P+i(1+i)^2P=(1+i)^3P\\
\vdots & \vdots\\
n & (1+i)^nP\\
\hline
\end{array}$$

\url{https://www.desmos.com/calculator/kqz7bwj8ip}

So we have the general equation $$A=P(1+i)^n,$$ so as usual, if you know any 3 of these values, in principle we should be able to find the 4th.\\

\subsection{AN ISSUE OF EXTREME ANNOYANCE}

The formulas we have found above is not so very complicated, and neither is solving for values found therein.  However, financial companies have devised a concept that obfuscates this principle via an artificial notion of ``compounding periods".\\

Suppose you have a debt of \$100 that accrues 10\% annual interest.  With no other given information, we would say that the amount you owe in $10$ years would be $$A=100(1.1)^{10}\approx \$259.37$$  right?  But what a financial institution will do is say that the debt is ``compounded quarterly" i.e. 4 times a year.  Thus instead of 10\% interest per year, it will now be $10\%/4=2.5\%$ interest \textbf{ per quarter}.  How much will you owe in 10 years?  Well, 10 years is 40 quarters so: $$A=100(1.025)^{40}\approx 268.51$$

Similarly, if the compounding is done  \textbf{ monthly}, then this is 12 times a year, and each month you'd be charged at a rate of $\frac{10\%}{12}$.  So in 10 years or 120 months, you would owe: $$A=100(1+\frac{0.1}{12})^{120}\approx 270.70$$

My belief is this accomplishes 2 things:

\begin{enumerate}
\item By making the issue more complicated, it makes it more difficult for consumers to figure out how much they are actually going to be charged in interest.
\item By increasing compounding periods, you can effectively charge more interest, see how much the debt's grow in the above scenarios?
\end{enumerate}

In general if you are charged an annual interest rate of $r$ compounded $m$ times a year for $t$ years.  You are really getting charged $$i=\frac{r}{m}$$ interest rate per time unit, and the number of time units will be $$n=mt.$$  How much yearly interest is there really?  It's how much interest you end up accruing in a year:  Let $r_E$ be the effective annual interest:

\begin{eqnarray*}
P(1+r_E)&=&P(1+r/m)^{m}\\
1+r_E&=&(1+r/m)^{m}\\
r_E&=&(1+r/m)^{m}-1\\
\end{eqnarray*}


So 10\% interest compounded quarterly is effectively:

$$r_E=(1+0.1/4)^4-1\approx0.1038$$ or 10.38\% annual interest.  Whereas 10\% compounded monthly is $$r_E=(1+0.1/12)^{12}-1\approx 0.1047$$ or 10.47\% annual interest.




\subsection{Solving}

Now that we got that annoyance out of the way, let's solve some example problems.

\begin{example}
Suppose that I take out a loan for \$500 at 12\% annual interest charged biannually.  How much will I owe in 10 years?\\

So we note that:

\begin{eqnarray*}
P&=&500\ \text{the initial loan}\\
i&=&0.12/2=0.06 \ \text{the interest rate per half-year}\\
n&=&10\cdot 2=20 \ \text{the number of half-years}\\
\end{eqnarray*}

So we are solving:

$$A=500(1+.06)^{20}\approx 1603.57.$$ \url{https://www.desmos.com/calculator/fuvteaqq2q}

\end{example}


\begin{example}
Suppose that I want to invest some money at 8\% annual interest compounded annually (so no shenanigans), and after 20 years I want \$2000.  How much should I invest?

So we note that:

\begin{eqnarray*}
A&=&2000, \ \text{the future value}\\
i&=&0.08, \ \text{the actual interest rate}\\
n&=&20, \ \text{number of years}
\end{eqnarray*}

So we are solving:

\begin{eqnarray*}
2000&=&P(1.08)^{20}\\
P&=&\frac{2000}{1.08^{20}}\approx429.10
\end{eqnarray*}

\url{https://www.desmos.com/calculator/2cavnhsskf}

\end{example}


\begin{example}
Suppose that I want to invest \$500 and in 10 years I want this to grow to \$2000.  What sort of interest would make that possible?\\

So we have:

\begin{eqnarray*}
P&=&500\\
A&=&2000\\
n&=&10
\end{eqnarray*}

So we are solving:

\begin{eqnarray*}
2000&=&500(1+i)^{10}\\
4&=&(1+i)^{10}\\
4^{1/10}=1+i\\
i&=&4^{1/10}-1\approx0.1487
\end{eqnarray*}

or 14.87\% annual interest.

\url{https://www.desmos.com/calculator/tzcuetbwgc}

\end{example}

\begin{example}
Suppose that I borrow \$1000 at 24\% annual interest compounded monthly.  How long until I owe \$20,000?\\

So we have:

\begin{eqnarray*}
P&=&1000\\
A&=&20000\\
i&=&0.24.12=0.02
\end{eqnarray*}

So we are solving:

\begin{eqnarray*}
20000&=&1000(1.02)^n\\
20&=&(1.02)^n\\
\ln(20)&=&n\ln(1.02)\\
n&=&\ln(20)/\ln(1.02)\approx151.28
\end{eqnarray*}

So in about 151-152 \textbf{ MONTHS}, I will owe 20,000 dollars.

\url{https://www.desmos.com/calculator/rpq1zumxyw}



\end{example}

\section{Regular Payments}\label{Section:RegularPayments}

\subsection{Future Value}

Let's say you have a back account that gives you 2\% monthly interest (lucky you!)  and let's also say that from January to May, you deposit \$100 each month.  How much will you have in May? Well:

\begin{itemize}
\item The 100 dollars you put in January would accrue interests for 4 months, so you'd have $100(1.02)^4$ dollars.
\item The 100 dollars you put in February would accrue interests for 3 months, so you'd have $100(1.02)^3$ dollars.
\item The 100 dollars you put in March would accrue interests for 2 months, so you'd have $100(1.02)^2$ dollars.
\item The 100 dollars you put in April would accrue interests for 1 months, so you'd have $100(1.02)^1$ dollars.
\item The 100 dollars you put in May would accrue interests for 0 months, so you'd have $100(1.02)^0$ dollars.
\end{itemize}

So you'd have: $$100(1.02)^4+100(1.02)^3+100(1.02)^2+100(1.02)^1+100(1.02)^0\approx \$520.40$$

But what if this went on for 10 months?  For 10 years?  You probably wouldn't want to calculate every single one of these.  So is there an easier way to compute something like $$100(1.02)^4+100(1.02)^3+100(1.02)^2+100(1.02)^1+100(1.02)^0?$$

So we have a sum of things that look like: $x^n+x^{n-1}+\cdots+x^1+x^0$.  There is a funny fact about this sum.  if we multiply it by $(x-1)$:

\begin{eqnarray*}
(x^n+x^{n-1}+\cdots+x^1+x^0)(x-1)&=&(x^n+x^{n-1}+\cdots+x^1+x^0)x-(x^n+x^{n-1}+\cdots+x^1+x^0)\\
&=&x^{n+1}+x^n+\cdots+x-x^n-x^{n-1}-\cdots-x^0\\
&=&x^{n+1}-x^0\\
&=&x^{n+1}-1
\end{eqnarray*}

Thus:

\begin{eqnarray*}
(x^n+x^{n-1}+\cdots+1)(x-1)&=&x^{n+1}-1\\
x^n+x^{n-1}+\cdots+1&=&\frac{x^{n+1}-1}{x-1}
\end{eqnarray*}


So, we should note that the future value of our investment above is:

\begin{eqnarray*}
100(1.02)^4+100(1.02)^3+100(1.02)^2+100(1.02)^1+100(1.02)^0&=&100((1.02)^4+(1.02)^3+(1.02)^2+(1.02)^1+(1.02)^0)\\
&=&100\frac{(1.02)^5-1}{1.02-1}\\
S&=&100\frac{(1.02)^5-1}{.02}
\end{eqnarray*}

Where $S$ is the future value.  Now, if we did this for $n$ months instead of 5, we would have:

$$S=100\frac{(1.02)^n-1}{.02}$$

If the monthly (or whatever time unit) interest rate was $i$ instead of 2\%, we would get:

$$S=100\frac{(1+i)^n-1}{i}$$

And if the payments were $R$ instead of \$100, we would have:

$$S=R\frac{(1+i)^n-1}{i}$$


So here we have the formula for the  future value of a debt or investment based on a constant payment.

\subsection{Present Value}

So if we have $n$ payments of $R$ at regular interest rate $i$, the future value is $S=R\frac{(1+i)^n-1}{i}$.  What is it's present value?  Well remember that the relationship between present and future value is:

\begin{eqnarray*}
S&=&P(1+i)^n\\
P&=&\frac{S}{(1+i)^n}.
\end{eqnarray*}
From this we get:

\begin{eqnarray*}
P&=&\frac{R\frac{(1+i)^n-1}{i}}{(1+i)^n}\\
&=&R\frac{((1+i)^n-1)\frac{1}{(1+i)^n}}{i}\\
&=&R\frac{1-\frac{1}{(1+i)^n}}{i}\\
P&=&R\frac{1-(1+i)^{-n}}{i}
\end{eqnarray*}

Suppose that we invest \$100,000 in an account that gives 1\% monthly interest and I'd like to pull money out for the next 10 years.  What is the Max that I could pull out?\\

Well, we have a present value of $P=100,000$, a monthly interest rate of $i=0.01$ and $n=120$ months.  So:

\begin{eqnarray*}
100000&=&R\frac{1-(1.01)^{-120}}{.01}\\
1000&\approx&R(1-0.302995)\\
R&\approx&\frac{1000}{0.697005}\\
&\approx&\$1434.71
\end{eqnarray*}

So with an investment like that, you can pull out \$1434.71 for the next 10 years.







