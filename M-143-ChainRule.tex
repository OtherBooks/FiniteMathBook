\documentclass[10pt]{article}

% The following command leaves more space between lines.  That's great
% when correcting drafts.  When you comment it out, however, the
% output looks much nicer.
%
\linespread{1.0}

\usepackage{amsmath}
\usepackage{amssymb}
\usepackage{graphicx}
\usepackage{epsfig}
\usepackage{latexsym}
\usepackage{amsthm}

\usepackage{mathrsfs}

\usepackage{multicol}



\usepackage[colorlinks,citecolor=blue]{hyperref}

\usepackage[latin1]{inputenc}

\usepackage{tikz-cd}
\usepackage{pgfplots}

%\usepackage{3dplot}

\usetikzlibrary{matrix,arrows,decorations.pathmorphing}


\usepackage[scale=0.8]{geometry}


%\usepackage{umoline}\setlength{\UnderlineDepth}{1pt}
%\usepackage[linktocpage=true]{hyperref}

\input xy
\xyoption{all}


%\addtolength{\hoffset}{-.5in}
%\addtolength{\textwidth}{1in}
%\setlength{\parindent}{.5in}
%\setlength{\textheight}{9.5in} \setlength{\topmargin}{-2cm}


\pagestyle{myheadings}\parindent 0em




\usepackage[latin1]{inputenc}


%------------------copy and posted code from the internets-------------

%\numberwithin{equation}{section} % comment out when neccessary

\newtheorem{theorem}[equation]{Theorem}
\newtheorem{lemma}[equation]{Lemma}
\newtheorem{proposition}[equation]{Proposition}
\newtheorem{corollary}[equation]{Corollary}


\theoremstyle{definition}
\newtheorem{definition}[equation]{Definition}
\newtheorem{example}[equation]{Example}
\newtheorem{remark}[equation]{Remark}
\newtheorem{problem}[equation]{Problem}



\newcommand{\R}[1]{\mathbb{R}^{#1}}
\newcommand{\C}[1]{\mathbb{C}^{#1}}
\newcommand{\Z}[1]{\mathbb{Z}^{#1}}
\newcommand{\K}[1]{\mathbb{K}^{#1}}
\newcommand{\embed}[0]{\hookrightarrow}
\newcommand{\TT}[4]{\begin{tabular}{| c | c |}\hline $#1$ & $#2$ \\ \hline $#3$ & $#4$ \\ \hline\end{tabular}} %goddamn it
\newcommand{\partd}[2]{\frac{\partial #1}{\partial #2}}
\newcommand{\limit}[2]{\displaystyle{ \lim_{#1 \to #2}}}
\newcommand{\vectornorm}[1]{\left|\left|#1\right|\right|}
\newcommand{\Ker}[0]{\text{\textnormal{Ker}}}
\newcommand{\Hom}[0]{\text{\textnormal{Hom}}}
\newcommand{\circled}[1]{\tikz[baseline=(char.base)]{
            \node[shape=circle,draw,inner sep=2pt] (char) {#1};}}


\newcommand{\T}{\rotatebox[origin=c]{180}{$\scriptscriptstyle \perp $}}
\newcommand{\x}{\textbf{x}}
\newcommand{\y}{\textbf{y}}
\newcommand{\supp}{\text{\textnormal{supp}}}
\newcommand{\csupp}{\text{\textnormal{cosupp}}}
\newcommand{\found}{\text{\textnormal{found}}}
\newcommand{\roof}{\text{\textnormal{roof}}}

\newcommand{\bcup}{\displaystyle\bigcup}
\newcommand{\bcap}{\displaystyle\bigcap}
\newcommand{\dsum}{\displaystyle\sum}
\newcommand{\dint}{\displaystyle\int}





\begin{document}
%

{\bf Name:} \hrulefill\hrulefill\hrulefill\\
{\bf M143} \qquad \qquad \\
{\bf Chain rule of Derivatives}\\ %(look familiar??)\\
%Show all work for full/partial credit.
%---------------- End of the document ---------------

\section{Chain Rule}

We've so far covered sums products and quotients of functions, so how about composition?   The {\bf chain rule} is: $\frac{d}{dx}[f(g(x))]=f'(g(x))g'(x)$.  To see why:

\begin{eqnarray*}
\frac{d}{dx}[f(g(x))]&=&\limit{h}{0}\frac{f(g(x+h))-f(g(x))}{h},\ \text{Similar to the product rule, we will multiply by 1:}\\
&=&\limit{h}{0}\frac{f(g(x+h))-f(g(x))}{h}\textcolor{blue}{\frac{g(x+h)-g(x)}{g(x+h)-g(x)}}\\
&=&\limit{h}{0}\frac{f(g(x+h))-f(g(x))}{g(x+h)-g(x)}\frac{g(x+h)-g(x)}{h}\\
&=&\limit{h}{0}\frac{f(g(x+h))-f(g(x))}{g(x+h)-g(x)}\limit{h}{0}\frac{g(x+h)-g(x)}{h}\\
\end{eqnarray*}

We're going tom play a second trick here, we are goin to let $k=g(x+h)-g(x)$, we also note then that $g(x+h+=g(x)+k$.  Also note that when $h\to 0$  It follows that $k=g(x+h)-g(x)\to g(x+0)-g(x)=0$.  So we can make some replacements:

\begin{eqnarray*}
&=&\limit{h}{0}\frac{f(g(x+h))-f(g(x))}{g(x+h)-g(x)}\limit{h}{0}\frac{g(x+h)-g(x)}{h}\\
&=&\limit{\textcolor{red}{k}}{0}\frac{f(\textcolor{red}{g(x)+k})-f(g(x))}{\textcolor{red}{k}}\limit{h}{0}\frac{g(x+h)-g(x)}{h}\\
&=&f'(g(x))g'(x)
\end{eqnarray*}

\section{Examples}

Consider $f(x)=\sqrt{x^2+1}$.  We can think of this as $f(x)=g(h(x))$ where $g(x)=\sqrt{x}=x^{1/2}, h(x)=x^2+1$.  So $g'(x)=\frac{1}{2}x^{-1/2}, h'(x)=2x$ and:

\begin{eqnarray*}
f'(x)&=&\frac{d}{dx}[\sqrt{x^2+1}]\\
&=&\frac{d}{dx}[g(h(x))]\\
&=&g'(h(x))h'(x)\\
&=&\frac{1}{2}(x^2+1)^{-1/2}(2x)\\
&=&\frac{x}{\sqrt{x^2+1}}.
\end{eqnarray*}


Another example, lets consider the derivative of $y=(x^2+100)^{300}$.  We COULD expand this polynomial out, but lets agree no one wants that.  On the other hand by the chain rule:

\begin{eqnarray*}
\frac{dy}{dx}&=&\frac{d}{dx}[(x^2+100)^{300}],\ \text{so by the chain rule:}\\
&=&300(x^2+100)^{299}\frac{d}{dx}[x^2+100]\\
&=&300(x^2+100)^{299}(2x)\\
&=&600x(x^2+100)^{299}.
\end{eqnarray*}






















\end{document}
